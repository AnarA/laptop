\label{page:siv}
Eine Universität ist ein lebendiger Ort! Die Uni verändert sich und ist in
Bewegung: Sie wächst hier und schrumpft vielleicht dort, sie verändert sich von
innen heraus und durch äußere Einflüsse, sie entwickelt Meinungen und
Stellungnahmen, und an all diesen Prozessen sind alle Mitglieder der
Universität beteiligt: Professoren, wissenschaftliche Mitarbeiter, technisches
und Verwaltungspersonal und -- als zahlenmäßig größte Mitgliedergruppe -- die
Studenten! Denn auch wir Studenten sind Teil der Universität und es liegt an
uns, sie zu verändern und zu gestalten! 

Die Möglichkeiten zur Mitgestaltung sind vielfältig: In dutzenden Ausschüssen
der Fakultäten und Fachbereichen sitzen studentische Vertreter stimm- und
gleichberechtigt mit den Professoren und Mitarbeitern zusammen und machen sich
zum Beispiel Gedanken über die künftige strategische Ausrichtung der Fakultät,
die Einführung neuer Studiengänge oder den Lehrveranstaltungsplan des kommenden
Semesters. In all diesen Ausschüssen wird Sacharbeit betrieben und zumindest
aus dem Fachbereich Mathematik lässt sich berichten, dass die Studenten dabei
durchweg als gleichberechtigte Partner wahr- und ernstgenommen werden. 

Auch auf Ebene der Universität gibt es Möglichkeiten zur studentischen
Mitbestimmung, die Gremien kümmern sich jedoch mehr um die \glqq großen\grqq\
Dinge wie beispielsweise das Leitbild und Selbstverständnis der Universität.
Zudem geht es auf Ebene der Universität normalerweise deutlich politischer zu
als in den Fakultäten und Fachbereichen: Hochschulpolitische Listen, den großen
Parteien nahe stehend, und unabhängige Listen ringen um die wenigen Plätze im
Akademischen Senat und den anderen Gremien.

Als weitere Säule neben der sachlich geprägten Gremienarbeit in den Fakultäten
gehört zur studentischen Interessensvertretung auch die Selbstverwaltung. Wie
in den meisten Bundesländern bilden auch in Hamburg die Studenten jeder
Hochschule eine sogenannte \emph{Verfasste Studierendenschaft}. Diese verfügt
über einen eigenen Etat und vertritt ihre Interessen selbst. An der Universität
Hamburg geschieht dies über das \emph{Studierendenparlament} (StuPa), welches
jährlich von allen Studierenden gewählt wird. Das StuPa wiederum bestimmt mit
der Mehrheit seiner Vertreter die Mitglieder des \emph{Allgemeinen
Studierendenausschusses} (AStA). Dieser fungiert als Regierung der Studierenden
und hat weitreichende Befugnisse. Daneben ernennt er eine gewisse Zahl von
Referenten für verschiedenen Themenbereiche (Hochschulpolitik, Soziales,
Kultur, \ldots) und bietet Sprechstunden und Beratungen an, zum Beispiel zu
Themen wie BAföG, Studium mit Kind, usw. Eure erste Anlaufstelle für Probleme
dieser Art sollte also der AStA sein. Zu finden ist er auf dem Campus,
gegenüber des WiWi-Bunkers, und unter \url{http://www.asta-uhh.de}.


Während die studentische Interessenvertretung auf dem Campus recht weit
entfernt zu sein scheint, findet ihr auch im Geomatikum eine zuständige
Interessenvertretung: Alle Studenten einer Fachrichtung bilden zusammen eine
Fachschaft. Diese wählt jedes Semester in einer Vollversammlung aus ihrer Mitte
den \emph{Fachschaftsrat} (FSR). Der FSR Mathematik ist also die für euch erste
Anlaufstelle, wenn es um die Vertretung eurer Interessen innerhalb des
Fachbereichs Mathematik geht. Neue Mitglieder sind immer gern gesehen, die
regelmäßigen Sitzungen sind offen für jeden Interessierten und jede Frage oder
Anregung. Näheres zu den Aufgaben und Aktivitäten des FSR findet ihr im Artikel
auf Seite \pageref{page:fsr}.

Es ist ausgesprochen reizvoll, als Student mitgestalten und -bestimmen zu
können. Die Möglichkeiten zur Einflussnahme sind zahlreich und wichtig für die
Situation der Studierenden: Welche Person soll die unbesetzte Professur
übernehmen? Wie können mehr studentische Arbeitsräume zur Verfügung gestellt
werden? 

Die Mitarbeit in den Gremien schult auch die sogenannten {\it soft
skills}: Selbstbewusstsein, Fähigkeit zu Kommunikation und Diskussion,
Teamwork, usw. Und schließlich knüpft man auf diese Weise sehr enge und
persönliche Kontakte zu den Professoren, was für die Gestaltung des eigenen
Studiums nur von Vorteil sein kann. 

Wer nun neugierig geworden ist und mehr wissen möchte, um vielleicht nach einem
Jahr des Ankommens seine Fühler noch weiter auszustrecken und das Gebilde
Universität besser und von einer ganz anderen Seite kennenzulernen, der schaue
einfach mal unverbindlich beim FSR vorbei.
\end{multicols}

\small
\begin{center}
\begin{tabular}{||p{24mm}||p{73mm}|p{73mm}|p{76mm}||}

\hhline{|t:=:t:=t=t=:t|} & {\bf Ohne Studenten}
    & {\bf Mit studentischer Beteiligung}
    & {\bf Studentische Selbstverwaltung} \\

\hhline{|:=::===:|} {\bf Gesamte Uni}
    & {\bf Hochschulrat}

      \begin{itemize}\itemsep 0pt\parskip 0pt
          \item Wahl des Uni-Präsidenten
          \item Genehmigung von Gebührensatzung, Struktur- und
                Entwicklungsplanung
      \end{itemize}
      {\bf Uni-Präsidium}

      \begin{itemize}\itemsep 0pt\parskip 0pt
          \item Presse- und Öffentlichkeitsarbeit
          \item zentrale Machtbefugniss, Verteilung der Finanzen
      \end{itemize}
    & {\bf Akademischer Senat (AS)}
      
      \begin{itemize}\itemsep 0pt\parskip 0pt
          \item höchstes Gremium, an dem alle Statusgruppen beteiligt sind
          \item stimmt der Wahl des Uni-Präsidenten zu
          \item Präsidium ist dem AS rechenschaftspflichtig
          \item Stellungnahme zu allen Themen in der Universität
      \end{itemize}
    & {\bf Studierendenparlament (StuPa) / AStA}

      \begin{itemize}\itemsep 0pt\parskip 0pt
          \item jährlich von allen Studenten der Universität gewählt
          \item Vertretung politischer, sozialer und kultureller Belange
      \end{itemize} \\
\hhline{||-||-|-|-||} {\bf MIN-Fakultät}
    & {\bf MIN-Dekanat}
      
      \begin{itemize}\itemsep 0pt\parskip 0pt
          \item steht der Fakultät vor, vertritt die Fakultät innerhalb der
                Universität
          \item verfügt über Finanzen der Fakultät, insb. Entscheidung über
                Ausschreibung von Stellen
          \item strategische Ausrichtung der Fakultät
      \end{itemize}

    & {\bf Fakultätsrat (FAR)}
      
      \begin{itemize}\itemsep 0pt\parskip 0pt
          \item berät und kontrolliert das Dekanat, insb. die Verwendung von
                Finanzen
          \item setzt Ausschüsse für Sacharbeit auf Fakultäts- und
                Fachbereichsebene ein
          \item berät über Prüfungsordnungen
      \end{itemize}

    & {\bf MIN-Aktiven-Treffen}

      \begin{itemize}\itemsep 0pt\parskip 0pt
          \item kein formales Gremium, sondern ein gelegentliches Treffen von
                in der Fakultät aktiven Studenten (FSRe)
          \item Informationsaustausch, Absprachen, Planung gemeinsamer
                Initiativen, etc.
      \end{itemize} \\
\hhline{||-||-|-|-||} {\bf Fachbereich Mathematik}
    & {\bf Fachbereichsleitung}

      \begin{itemize}\itemsep 0pt\parskip 0pt
          \item vertritt den Fachbereich in der Fakultät und Universität
          \item verfügt über die Finanzmittel des Fachbereichs
          \item enge Abstimmung mit den Mitgliedern des Fachbereichs
      \end{itemize}

    & {\bf Ausschüsse des FAR}, z.B.

      \begin{itemize}\itemsep 0pt\parskip 0pt
          \item Fachausschuss (FA) Mathematik
          \item Bachelor- und Master-Ausschüsse
          \item Prüfungsausschüsse
          \item Berufungsausschüsse
      \end{itemize}

      {\bf Informelle Kommissionen}, z.B.

      \begin{itemize}\itemsep 0pt\parskip 0pt
          \item Entwicklung des Vorkurses
          \item Jahr der Mathematik
      \end{itemize}

    & {\bf Fachschaftsrat (FSR)}

      \begin{itemize}\itemsep 0pt\parskip 0pt
          \item von allen Studenten der Mathematik gewählt
          \item Vertretung der Interessen vor Ort
          \item guter Kontakt zu Professoren und Fachbereichsleitung
          \item Service (Beratung, Prüfungsprotokolle)
          \item Information (Hochschulpolitik, intern)
          \item Vernetzung im Fachbereich
          \item Freizeit (Kartenspielen, Weihnachtsfeier, Parties, etc.)
      \end{itemize} \\
\hhline{|b:=:b:===:b|}
\end{tabular}
\end{center}
\begin{multicols}{2}

\normalsize

Dieses Schaubild soll euch einen Überblick über die verschiedenen, relevanten
Gremien auf Universitäts-, Fakultäts- und Fachbereichsebene geben. Näheres zu
den Gremien und ihren Zuständigkeiten erfahrt ihr in der Einheit Studentische
Interessenvertretung. Sollten noch Fragen offenbleiben, kann euch der FSR
weiterhelfen, der sich über Besuch immer sehr freut. Die regelmäßigen Sitzungen
sind öffentlich.
