%\documentclass{amsart}
%\usepackage{amssymb}

\subsection{Lineare Algebra \& Analytische Geometrie}

\bigskip \noindent Mathematik habe ich an der Universit\"at Bonn studiert und
dort auch 2000 promoviert. W\"ahrend meiner Assistentenzeit war ich bei
l\"angeren Forschungsaufent\-halten in Stra{\ss}burg und Oslo. Seit 2005 bin
ich Professorin f\"ur Algebraische Topologie in Hamburg. Meine
Forschungsschwerpunkte liegen in der stabilen Homoto\-pie\-theorie und der
homologischen Algebra.

\medskip

Die Lehre liegt mir sehr am Herzen und ich freue mich darauf, Ihnen in der
Vorlesung \emph{Lineare Algebra und analytische Geometrie} einen ersten
Einblick in die faszinierende Welt der Mathematik geben zu k\"onnen!


\medskip

Die Lineare Algebra, wie auch die Analysis, geh\"ort zu den Grundlagen eines
jeden Mathematikstudiums. Die Inhalte dieser Vorlesung werden Sie in Ihrem
weiteren Studium brauchen. Dar\"uberhinaus erlernen Sie in den Grundvorlesungen
die Arbeitsmethoden, die Sie als Mathematiker und Mathematikerin ben\"otigen,
sei es in der Schule, in der Wirtschaft oder in der Forschung. 

\medskip

Mathematik lernt man, indem man selbst Mathematik macht. Seien Sie aktiv und
verfallen Sie in den Vorlesungen bitte nicht in eine Konsumhaltung!  Nur das,
was Sie sich selbst angeeignet haben, verstehen Sie auch. Bleiben Sie
neugierig! Wir bieten Ihnen \"Ubungsaufgaben an, um Ihnen bei der
Auseinandersetzung mit dem Stoff der Vorlesungen zu helfen. Aufgaben zu
bearbeiten ist zwar anstrengend, aber es lohnt sich.  Mathematik macht
gl\"ucklich: Wenn Sie lange an einem Problem geknobelt haben und Sie haben es
schlie{\ss}lich verstanden, dann ist das ein gutes Gef\"uhl. 

Nutzen Sie bitte die Angebote, die wir bieten. Wenn Sie unvorbereitet in die
\"Ubungsgruppen gehen, werden Sie kaum etwas mitnehmen. Wenn Sie die Vorlesung
nacharbeiten, werden Sie sicherlich Fragen haben. Stellen Sie diese in den
\"Ubungsgruppen und in den Tutorien. Haken Sie solange nach, bis Sie alles
verstanden haben! 

\medskip

Ich hoffe, Ihnen gelingt die Umstellung von der Schule auf die Universit\"at
und von der Schulmathematik auf Mathematik.   F\"ur Ihren Studienbeginn
w\"unsche ich Ihnen viel Erfolg und alles Gute!

\bigskip

\hfill Birgit Richter

\bigskip \noindent
Birgit Richter \\
Bereich Algebra und Zahlentheorie\\
Telefon: 42838 - 5173, Sekretariat: 5171\\
Email: birgit.richter@uni-hamburg.de\\
http://www.math.uni-hamburg.de/home/richter/
