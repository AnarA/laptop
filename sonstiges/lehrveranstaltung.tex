\begin{tabularx}{\textwidth}{|l|X|X|X|}
% I. Bachelor (Mathematik, Wirtschaftsmathematik, Mathematik Lehramt an Gymnasien und Lehramt an Beruflichen Schulen)
\hline 65--001 	& Orientierungseinheit f"ur Studienanf"anger/innen (Mathematik und Wirtschaftsmathematik Bachelor, alle Lehr"amter mit Mathematik als Unterrichtsfach)               
								& Block\-ver\-an\-stal\-tung vom 05.-14.10.11; 4,57 SWS, ganzt"agig, Beginn: 05.10.11, 09:30 Uhr Geom H1               
								& Prof. Dr. Armin Iske, Libor Kadrnka \\
\hline 65--002 	& Tutorium f"ur ausl"andische Studierende (insbesondere im Rahmen von ERASMUS)               
								& 2 SWS, Zeitpunkt noch unbekannt
								& Prof. Dr. Armin Iske \\
								
\hline 65--003 	& Vorkurs Mathematik (Blockveranstaltung vom 26.09 - 04.10.11)
           			& 2,43 SWS, Mo.-Do. 09:30-11:00, 11:30-13:00, Fr. 08:30-10:00, 10:30-12:00, Mo.-Fr. 17:00-18:15, Di. 04.10. 16:15-17:30, Geom H1, H2, Beginn 26.09.2011
								& Dr. Anna Posingies, Jan Steffen M"uller \\
								
\hline 65--004 	& "Ubungen zu Vorkurs Mathematik (Blockveranstaltung vom 26.09.-04.10.11) (7 Gruppen)
								& 0,86 SWS, Mo.-Fr. 15:15-16:45, Di. 04.10. 14:15-15:45, Geom 241, 430, 431, 432, 434, 435, 1241, H3
								& N.N. \\
\hhline{|=|=|=|=|} 65--011 & Lineare Algebra und Analytische Geometrie I
	                         & 4 SWS, Mo. 14:15-15:45, Mi. 08:15-09:45, Geom H1, Beginn: 17.10.11
	                         & Prof. Dr. Mathias Schacht \\
								
\hline 65--012 	& "Ubungen zu Lineare Algebra und Analytische Geometrie I (10 Gruppen)
								& 2 SWS, Di. 14:15-15:45, Geom 241, 431, 435, Di. 16:15-17:45, Geom 434, Mi. 10:15-11:45, Geom 434 (\& 431, 435 f"ur LG/LBS), Mi. 12:15-13:45, Geom 431, 434, 435, Beginn: 18./19.10.11
								& N.N. \\
								
\hline 65--014 	& Tutorium zu Lineare Algebra und Analytische Geometrie I (9 Gruppen)
								& 1 SWS, Mo. 12:15-13:00, Geom 434, 435 (\& 431 f"ur LG/LBS), Mo. 13:15-14:00, Geom 435 (\& 431 f"ur LG/LBS), Di. 12:15-13:00, Geom 241, 435, Di. 13:15-14:00, Geom 241, 435, Beginn: 17./18.10.11
								& Matthias Hamann, N.N. \\
								
\hline 65--021 	& Analysis I
								& 4 SWS, Mi. 14:15-15:45, Fr. 14:15-15:45 Geom H1, Beginn: 19.10.11
								& Prof. Dr. Janko Latschev \\
								
\hline 65--022 	& "Ubungen zu Analysis I (9 Gruppen)
								& 2 SWS, Mi. 16:15-17:45 Geom 431, 435 (LG/LBS), Do. 08:15-09:45, Geom 241, 1241, Do. 10:15-11:45, Geom 241, 431, 1241, Fr. 08:15-09:45, Geom 241, 434, Beginn: 19./20./21.10.11         
								& N.N. \\
								
\hline 65--024 	& Tutorium zu Analysis I (5 Gruppen)
								& 2 SWS, Do. 08:15-09:45, Geom 431, 435, Do. 12:15-13:45, Geom 431, 435, Do. 14:15-15:45, Geom 431, Beginn: 20.10.11
								& Dr. Ernst B"onecke, N.N. \\
\hline
\end{tabularx}
\begin{tabularx}{\textwidth}{|l|X|X|X|}
%Keine Markierung, welche LG/LBS sind.								
% II. Bachelor (Mathematik Lehramt der Primarstufe und Sekundarstufe I sowie Lehramt an Sonderschulen)
\hline 65--301 	& Grundlagen der Mathematik
								& 4 SWS, Mo. 14:15-15:45, Mi. 08:15-09:45, Geom H2, Beginn: 17.10.10
								& Dr. Susanne Margret Koch \\
\hline 65--302 & "Ubungen zu Grundlagen der Mathematik (5 Gruppen)
               & 2 SWS, Mo. 16:15-17:45, Geom 241, 431, 435, 1241, Di. 14:15-15:45, Geom 1241, Beginn: 17./18.10.11
               & Dr. Susanne Margret Koch, N.N. \\
% Enthalten oben.
%\hline 65-252 & "Ubungen zu Grundlagen der Mathematik f"ur Studierende der Lehr"amter Primarstufe und Sekundarstufe I sowie Sonderschulen (5 Gruppen)
%              & 2st., Mo 16.15-17.45 Geom 431,434,435,1241,Di 14.15-15.45 Geom 1241 Beginn: 19.10.10
%              & Hans-Joachim Samaga\\
% Ist aktuell. (Stand 18.9.2010)
\hline 65--324 & Arbeitsgruppenbetreuung zu Grundlagen der Mathematik, Grundbildung Analysis und Grundbildung Stochastik (5 Gruppen)
               & 4 SWS, Do. 18:00-21:00, Geom 430, 431, 432, 434, 435, Beginn: 20.10.11
               & Dr. Hubert Kiechle, Dr. Susanne Margret Koch \\
%entf"allt.
%\hhline{|=|=|=|=|} 65--902	& Geschichte der Mathematik
%	                          & 2 St., Fr. 12.15 - 13.45, Geom H1, Beginn: 22.10.10
%	                          & Prof. Dr. Thomas Sonar \\
%	                          
%\hline 65--922	 & Seminar zur Vorlesung 65--902: Geschichte der Mathematik
%	               & 2 St., Fr. 14:15 - 15:45, Geom E11/13, Beginn: 22.10.10
%	               & Prof. Dr. Thomas Sonar \\
	               
% III. Master (Mathematik Lehramt der Primarstufe und Sekundarstufe I sowie Lehramt an Sonderschulen)
%\hline 65--303 & Frauen in der Geschichte der Mathematik
%               & 2 St., Do. 14:45 - 15:45, Geom H4, Beginn: 21.10.10
%               & Prof. Dr. Andrea Blunck \\
%
%\hline 65--304 & "Ubungen zu Frauen in der Geschichte der Mathematik (2 Gruppen)
%               & 1 St., Do. 12:15 - 13:45, Geom H3, Beginn: 28.10.10
%               & Prof. Dr. Andrea Blunck \\
% IV, V, VI, VII fehlen! (TODO)
% Nebenfach: Physik
\hline 66--100 	& Physik I
                & 4 SWS, Di. 14:00-15:30, Do. 09:00-10:30, Jungius 9, H"ors II, Beginn: 25.10.11
                & Prof. Dr. Markus Drescher, Prof. Dr. Ulrich Merkt \\
                
\hline 66--101 	& Einf"uhrung in die Theoretische Physik I
	              & 3 SWS, Di. 15.45-16.45, Do. 10.45-12.00, Jungius 9, H"ors II, Beginn: 25.10.11
	              & Prof. Dr. Peter Schmelcher \\
	              
\hline 66--102 	& "Ubungen zur Physik I und Einf"uhrung in die Theoretische Physik I              
								& 3 SWS, Do. 13.00-15.15, 15.30-17.45, 18:00-20:15, Jungius 9, Bibliothek AP, SemRm 3, 4, 5, 6, Beginn: 03.11.11
								& N.N. \\
								
\hline 66--103	& Tutorium zur Physik I und Einf"uhrung in die Theoretische Physik I
	              & 2 SWS, keine Angaben (bei Studienbeginn unter www.physnet.uni-hamburg.de einsehbar)
	              & Prof. Dr. Markus Drescher, Prof. Dr. Ulrich Merkt, Prof. Dr. Peter Schmelcher \\
	              
% Nebenfach: Informatik
\hhline{|=|=|=|=|} 64--000 & Vorlesung Softwareentwicklung I (Modul IP1)
                           & 2 SWS, Mi. 14:15-15:45, H"orsaal A Chemie, Beginn: 19.10.11
                           & Dr. Axel Schmolitzky \\
                           
\hline 64--001 & "Ubungen zu Softwareentwicklung I (Modul IP1) (7 Gruppen)
               & 2 SWS, Mo. 09:00-12:00, Di. 09:00-12:00, 14:00 - 17:00, Do. 14:00-17:00, Fr. 09:00-12:00, D-010, D-017, D-018, Mi. 09:00-12:00, Do. 14:00-17:00, D-010, D-017, D-018, D-114, Beginn: 17./18./19./20./21.10.11
               & Dr. Axel Schmolitzky, Christian Sp"ah, Till Aust, Eugen Reiswich, Isabelle Streicher, Fredrik Winkler, Claudia Wyrwoll \\
               
\hline 64-070 & Vorlesung Algorithmen und Datenstrukturen (Modul IP4)
              & 3 SWS, Mi. 10:15-11:45, Erzwiss H, Fr. 14:15-15:45, Phil A, Beginn: 19.10.11
              & Prof. Dr. Matthias Rarey \\
\hline 64-071 & "Ubungen zu Algorithmen und Datenstrukturen (Modul IP4) (5 Gruppen)
              & 1 SWS, Mi. 14:15-15:45, ZBH Rm 16, F-635, F-534, Mi. 16:15-17:45, D-129, F-534, Do. 08:15-09:45, D-220, Do. 10:15-11:45, F-534, F-334, Fr. 12:15-13:45, ZBH Rm 16
              & Lennart Heinzerling, Matthias Hilbig, N.N. \\
              
% Gibt es nicht mehr.
%\hline 64-080 & Vorlesung Grundlagen von Datenbanken (GDB) (Modul IP5) 
%              & 3st. Mi 10:15-11:45, MLK6 HsB Chem Di 14:15 15:45, VMP8 Erzwiss H, Beginn: 21.10.09 
%              & Norbert Ritter \\
%
%\hline 64-081 & "Ubung zu Grundlagen von Datenbanken (Modul IP5)
%              & 1st. Mo 10:15-11:45 Mo 12:15-13:45 Mo 14:15-15:45 Di 8:15-9:45 Di 10:15-11:45, Beginn 19.10.09
%              & Norbert Ritter, Marc Holze, Fabian Panse, Michael von Riegen \\
              &&&\\
\hline
\end{tabularx}

F"ur Informatik waren dies nur einige der zu empfehlenden Veranstaltungen
Weitere Informatik-Vorlesungen siehe
\url{www.informatik.uni-hamburg.de/Info/Studium}.

\begin{tabularx}{\textwidth}{|l|X|X|X|}
% Nebenfach: VWL / BWL?
\hline 21-10.010 & Grundlagen des Rechnungswesens Z1/Z2
                             & 3 SWS, Di. 09:00 - 12:00, Audimax 1, Beginn: 18.10.11
                             & Dr. Andreas Mammen \\
\hline 21-10.010 & "Ubungen zu Grundlagen des Rechnungswesens Z1/Z2  (10 Gruppen)
                 & 1 SWS, Di. 08:00-09:00, WiWi 2095/2197, Di. 14:00-15:00, 15:00-16:00, WiWi 2095/2197, 2101/2105, Mi. 08:00-09:00, MBA H"orS 233, Mi. 14:00-15:00, WiWi 2163/2168, Do. 08:00-09:00, 09:00-10:00, MBA H"orS 233, Fr. 11:00-12:00, WiWi 2095/2197, Beginn: 25./26./27./28.10.11
                 & N.N. \\
                 
\hline 21-10.011 & Grundlagen des Rechnungswesens Z3/Z4
                 & 3 SWS, Mo 10:00-13:00, Audimax 1, Beginn: 17.10.11
                 & Dr. Ralf Wi"smann \\
                
\hline 21-10.011 & "Ubungen zu Grundlagen des Rechnungswesens Z3/Z4 (10 Gruppen)
                 & 1 SWS, Mo. 08:00-09:00, 09:00-10:00, WiWi 2101/2105, Mo. 13:00-14:00, 14:00-15:00, WiWi 3136/3142, Di. 16:00-17:00, 17:00-18:00, WiWi 2095/2197, Do 09:00-10:00 WiWi 2091/2201, Do. 13:00-14:00, WiWi 2067/2071, Fr. 10:00-11:00, 11:00-12:00, WiWi 2067/2071, Beginn: 24./25./27./28.10.11
                 & N.N. \\
\hline 21-10.020 & Grundlagen der Wirtschaftsinformatik Z1/Z2
                 & 2 SWS, Mo. 10:00-12:00, ESA B, Beginn: 17.10.11
                 & Dr. Gabriele Schneidereit \\
\hline 21-10.020 & "Ubungen zu Grundlagen der Wirtschaftsinformatik Z1/Z2 (6 Gruppen)
                 & 2 SWS, Do. 10:00-12:00 WiWi 2054/2055, 2098/2194, 2095/2197, 3136/3142, Fr. 14:00-16:00 WiWi 2067/2071, Fr. 16:00-18:00 WiWi 2067/2071, Beginn: 27./28.10.11
                 & Dr. Gabriele Schneidereit \\
                 
\hline 21-10.021 & Grundlagen der Wirtschaftsinformatik Z3/Z4
                 & 2 SWS, Di. 10:00-12:00, Erzwiss H, Beginn: 18.10.11
                 & Dr. Kai Br"ussau, Dr. Gabriele Schneidereit \\
                 
\hline 21-10.021 & "Ubungen zu Grundlagen der Wirtschaftsinformatik Z3/Z4 (6 Gruppen)
                 & 2 SWS., Di. 08:00-10:00, WiWi 0079, 2054/2055, Mi. 10:00-12:00, WiWi 2175/2181, 2101/2105, Mi. 16:00-18:00, WiWi 2054/2055, 2175/2181, Beginn: 25./26.10.11
                 & Dr. Kai Br"ussau, Dr. Gabriele Schneidereit \\
                 
% Nebenfach: TUHH? (TODO)
%\hhline{|=|=|=|=|} TUHH & Grundlagen der Elektrotechnik I
%                        & Donnerstag, 11:30-13 in H - SBS95 Raum Audimax1
%                        & G"unter Ackermann \\
%
%&&&\\
%
%\hline TUHH & Elektrotechnik I / Grundlagen der Elektrotechnik I
%            & Montag, 8:00-9:30 in I - DE22 Raum Audimax2, Donnerstag, 11:30-12:45 in I - DE22 Raum Audimax2
%            & Jan-Luiken ter Haseborg \\
%
%\hline TUHH & Technische Mechanik I
%            & Mittwoch, 8:00-9:30 in I - DE22 Raum Audimax2
%            & Uwe Weltin \\
\hline
\end{tabularx}

\begin{tabularx}{\textwidth}{|l|X|X|X|}
\hline 21-10.095 & Makro"okonomik Z1/Z2
                           	 & 3 SWS, Do. 8:00-11:00, Audimax 1, Beginn: 20.10.11
                             & Prof. Dr. Michael Funke \\
                            
\hline 21-10.095 & "Ubungen zu Makro"okonomik (15 Gruppen)
                 & 1 SWS, Mo. 08:00-09:00, 09:00-10:00, 14:00-15:00, 15:00-16:00, WiWi 2175/2181, Di. 08:00-09:00, 09:00-10:00, WiWi 2091/2201, Mi. 10:00-11:00, 11:00-12:00, 12:00-13:00, 13:00-14:00, WiWi 2091/2201, Fr. 10:00-11:00, WiWi 2101/2105, Fr. 11:00-12:00, 12:00-13:00, WiWi 2054/2055, 2101/2105, Beginn: 24./25./26./28.10.11
                 & N.N. \\
\hline 21-10.096 & Makro"okonomik Z3/Z4
                 & 3 SWS, Fr. 8:00-11:00, Audimax 1, Beginn: 21.10.11
                 & Dr. Sven Schreiber \\
\hline 21-10.096 & "Ubungen zu Makro"okonomik (12 Gruppen)
                 & 1 SWS, Mo. 13:00-14:00, WiWi 2101/2105, Di. 08:00-09:00, WiWi 2098/2194, Mi. 08:00-09:00, 09:00-10:00, WiWi 2054/2055, 2091/2201, Do. 08:00-09:00, 09:00-10:00, WiWi 0079, Do 10:00-11:00, 11:00-12:00, WiWi 2091/2201, Fr. 11:00-12:00, 12:00-13:00, WiWi 2175/2181, Beginn: 24./25./26./27./28.10.11
                 & N.N. \\
                 
\hline 21-10.110 & Investition Z1/Z2
                 & 2 SWS, Do 16:00-18:00, ESA A, Beginn: 20.10.11
                 & Prof. Dr. Alexander Bassen \\
                 
\hline 21-10.110 & "Ubungen zu Investition Z1/Z2 (9 Gruppen)
                 & 2 SWS, Mo. 15:00-17:00, WiWi 2095/2197, 2091/2201, Di. 10:00-12:00, WiWi 2101/2105, 2091/2201, Mi. 08:00-10:00, WiWi 2067/2071, Mi. 12:00-14:00, WiWi 2095/2197, 2098/2194, Fr. 08:00-10:00, WiWi 2163/2168, Fr. 10:00-12:00, WiWi 2163/2168, Beginn: 24./25./26./28.10.11
                 & N.N. \\
\hline 21-10.111 & Investition Z3/Z4
                 & 2 SWS, Do. 14:00-16:00, ESA A, Beginn: 20.10.11
                 & Prof. Dr. Alexander Bassen \\
\hline 21-10.111 & "Ubungen zu Investition Z3/Z4 (6 Gruppen)
                 & 2 SWS, Mo. 08:00-10:00, WiWi 2098/2194, Mo. 10:00-12:00, WiWi 2098/2194, Mo. 12:00-14:00, WiWi 2067/2071, Mi. 08:00-10:00, WiWi 2095/2197, Mi. 10:00-12:00, WiWi 2095/2197, B2, Beginn: 19./24./26.10.11
                 & N.N.\\
\hline
\end{tabularx}
