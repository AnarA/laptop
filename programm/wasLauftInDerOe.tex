Bevor du erfährst, was in der OE passiert, erst einmal ein kurzer Hinweis, wie
die OE abläuft. Die OE findet nicht parallel zu deinen ersten
Mathematik-Vorlesungen statt, sondern davor. Sie ist fester Bestandteil der
Anfängerausbildung eines Mathematikers.  Auch Wirtschaftsmathematiker brauchen
keine Angst zu haben, dass sie etwas versäumen. Außerdem ist die
Orientierungseinheit so geplant, dass Studenten des Lehramts die ersten drei
Tage an der Orientierungseinheit des Fachbereich Mathematik und anschließend an
der Orientierungseinheit der Erziehungswissenschaften teilnehmen können, ohne
etwas Wichtiges zu verpassen.

Die OE ist in Arbeitseinheiten gegliedert, was sich vielleicht etwas bedrohlich
anhört, aber ganz lieb gemeint ist und unter Umständen ganz lustig werden kann
(womit nicht gesagt sein soll, dass den Themen der ernste Hintergrund fehlt).
In jeder Arbeitseinheit soll, zumeist in Kleingruppen, ein bestimmter Aspekt
des (Mathematik-)Studiums behandelt und dir näher gebracht werden.

\subsection*{Studienberatung}

% TODO: Issue 4

Wie dein Studium aufgebaut ist, welche Veranstaltungen du wann besuchst, wie du
am meisten aus den Veranstaltungen herausholst, wann gute Zeitpunkte für
Prüfungen sind und viele andere Informationen zum Studium und um das Studium
herum erhältst du in der Arbeitseinheit \emph{Studienberatung}. Alles, was
direkt mit den ersten Semestern zu tun hat, wird im ersten Teil am Mittwoch,
den 05.10., behandelt.

Nur im Zusammenhang mit dem Rest der OE erfüllt die Teilnahme an dieser Einheit
die Anforderung des \S 51, Absatz 2 des Hamburgischen Hochschulgesetzes,
welcher besagt, dass jeder Student an einer Studienfachberatung teilgenommen
haben muss.

\subsection*{Lehr- und Lernformen}

An der Schule war alles so schön einfach: Der Lehrer kommt rein, erzählt ein
bisschen was -- und geht wieder raus. Meist sollte man dann zuhause zwei, drei
Aufgaben rechnen oder einige Seiten lesen. Das war's. Und verstanden habe ich
eh alles, ich bin ja nicht blöd -- und in Mathe sogar gut!

An der Uni ist das ein bisschen anders: Es gibt Vorlesungen, Tutorien,
Übungsgruppen und noch einige Formen mehr. Was das ist? Nun, genau das wollen
wir dir zeigen, indem wir einen typischen Zyklus mit euch durchspielen: eine
Vorlesung, ein Tutorium, eine Übungsgruppe und zwei Arbeitsgruppen, in denen
wir mit dir die Vor- und Nachbereitung von Vorlesungen und das Lösen von
Aufgaben üben wollen.

Nein, wir halten dich nicht für völlig doof, aber an der Uni ist eben alles
etwas anders: Du wirst nicht mehr alles \glqq einfach so\grqq\ verstehen. Und
was dann? Wem stelle ich welche Fragen? Wozu brauche ich den
Übungsgruppenleiter? Was ist überhaupt der Unterschied zwischen Professor und
Dozent? Wie arbeite ich vor und nach, wie arbeite ich effizient in einer
Gruppe? Auf fast alle diese Fragen hat man eine Antwort parat, und fast immer
ist sie falsch.

Diese Einheit soll euch den Umgang mit diesen Problemen zeigen, ohne dass ihr
den bitteren Weg des {\it trial and error} gehen müsst.

\subsection*{Q.E.D.}

Für einen erleichterten Einstieg in das erste Semester haben wir diese Einheit
ins Leben gerufen. Hier wollen wir mit dir die grundlegenden Dinge zum
korrekten Aufschreiben von Übungsaufgaben durchgehen, welche du dann auch
gleich für die Übungsaufgaben, die in der Einheit Lehr- und Lernformen gestellt
werden, nutzen kannst.

Diese Einheit umfasst einerseits einen Überblick über Beweismethoden, welche
wir auch gleich an einigen Beispielen üben wollen, andererseits einiges über
Logik und die Formulierung logischer Aussagen mithilfe von Junktoren, Quantoren
und Verknüpfungszeichen, welche dir in den ersten Semestern zahlreich begegnen
werden. Was sich hinter all diesen Begriffen versteckt, und dass dieses eher
hilfreiche als furchteinflößende Konzepte sind, wirst du hier lernen. (Und
auch, was \glqq q.e.d.\grqq\ bedeutet.)

Dabei soll die Einheit kein Versuch sein, den Besuch des ersten Semesters zu
ersetzen (das wäre in der kurzen Zeit auch gar nicht möglich). Wohl aber
möchten wir versuchen, dir den Einstieg zu erleichtern und insbesondere den
mühsamen Prozess, das korrekte Formulieren mathematischer Sachverhalte zu
lernen, ein wenig zu vereinfachen.

\subsection*{Studentische Interessenvertretung}

Hinter dem sperrigen Begriff der \emph{Studentischen Interessenvertretung}
(SIV) verbirgt sich die einfache Idee, dass zum einen für alle Formen von
Problemen zuständige Anlaufstellen existieren, zum anderen die Universität von
der Mitbestimmung und gemeinsamen Gestaltung durch alle ihre Mitglieder lebt.
Hier gibt es für dich als Student vielfältige Möglichkeiten, dich zu engagieren
-- sowohl allgemein (hochschul-)politisch als auch auf fachlicher Ebene in den
einzelnen Fakultäten und Departments.

% TODO: Issue 4

Alles dieses wollen wir am Montag, den 10.10. in der Einheit SIV anhand einiger
Beispiele genauer betrachten. Weitere Informationen findest du auch im
SIV-Artikel auf Seite \pageref{page:siv}.

\subsection*{Mathematik im Beruf}

Die wenigsten von euch, die Lehrämter ausgenommen, haben wohl schon konkrete
Vorstellungen davon, welchen Beruf sie ergreifen wollen und inwiefern sie das
Studium darauf vorbereitet. Um denen einen Einblick zu ermöglichen, welche
beruflichen Möglichkeiten es für Mathematiker und Wirtschaftsmathematiker gibt,
laden wir Personen aus verschiedenen Branchen ein, die uns von ihrem
Berufsalltag erzählen und für Fragen zur Verfügung stehen.

Einen kleinen Einblick in die Nützlichkeit der Mathematik im heutigen Leben
gibt auch folgende Seite des Fachbereich:
\url{http://www.math.uni-hamburg.de/teaching2/Studieninteressierte/studieninteressierte.html}

\subsection*{Stadtführung}

% TODO: Issue 4

Nicht nur für die Nicht-Hamburger ist dieser Programmpunkt gedacht.  Unter
Leitung unserer höchstkompetenten Tutoren wird unsere wunderschöne Stadt von
einer vielleicht ganz neuen Seite beleuchtet, und auch \glqq Eingeborene\grqq\
werden noch das eine oder andere neu entdecken; auf der \emph{Stadtführung} am
Montag, 10.10. um 15:30 Uhr.

\subsection*{Mathematik und Gesellschaft}

Hält man sich längere Zeit unter Mathematikern auf, so kann man schnell zu dem
Eindruck gelangen, sie seien allesamt Wesen aus einer anderen Welt: Sie
sprechen eine eigene Sprache, lachen über eigene Witze und scheinen oftmals in
Gedanken in viel höheren Sphären zu schweben als in den Niederungen des
Alltags.  Und man kann sich die Frage stellen, ob Mathematiker und die
Mathematik überhaupt einen Bezug zu dieser Welt haben.


Um dich ein wenig auf dem Boden der Tatsachen zu halten, möchten wir diese
Frage zusammen mit dir in der Einheit \emph{Mathematik und Gesellschaft} (M\&G)
untersuchen. Auch im Artikel \fpageref{page:mug} finden sich weitere Informationen und
Gedanken zu diesem Thema. Denn gerade im Verlauf der ersten Semester kann es
manchmal ganz hilfreich sein, sich daran zu erinnern, dass man auch nur ein
menschliches Wesen und von dieser Welt ist.

\subsection{Erasmus}

% TODO: Erasmus

\emph{ERASMUS} ist ein europäisches Austauschprogramm, das Studierenden
ermöglicht, einmalig bis zu 10 Monaten an einer europäischen Partneruniversität
zu studieren.  Insbesondere fördert ERASMUS die Mobilität von Studierenden und
Dozenten in Europa.  Neben der Erweiterung der fachlichen Qualifikation soll
der Auslandsaufenthalt vor allem fremdsprachliche und interkulturelle Kompetenz
fördern. Studierende lernen dabei eine andere Perspektive kennen und sammeln
wertvolle persönliche Erfahrungen.  Der Fachbereich Mathematik besitzt
ERASMUS Kooperationen mit mathematischen Fachbereichen aus 24
Partneruniversitäten in 11 europäischen Ländern.

Neugierig geworden? Weitere Informationen zum ERASMUS Programm, zu unseren
ERASMUS Partnern sowie zu den Teilnahmevoraussetzungen findet man auf der
Homepage \url{http://www.math.uni-hamburg.de/erasmus/}.

\subsection*{Abschlussfahrt nach Brahmsee}

% TODO: Issue 4

Am Ende der Orientierungseinheit, vom 13. bis 14. Oktober, fahren wir mit euch
auf eine Abschlussfahrt nach Brahmsee. So kurz vor Studienbeginn wollen wir
noch zwei nette Tage mit euch verbringen. Am Donnerstag gibt es ein kleines
Tagesprogramm, sowie einen supercoolen \emph{Bunten Abend}, den wir schließlich
gemütlich am Lagerfeuer ausklingen lassen wollen. 

% TODO: Issue 4

% TODO: Preis

Während der ersten Tage geben alle Tutoren Listen herum, auf denen du bis
spätestens Montag, den 10.10., eintragen kannst, ob du mitkommst und
gegebenfalls ein Auto zur Verfügung hast, in dem noch Plätze frei sind. Die
Fahrt kostet \EUR{25} pro Person, Fahrer bekommen jedoch eine großzügige
Ermäßigung für jeden Mitfahrer. Damit deine Anmeldung perfekt ist, möchten wir
dich bitten, die Teilnahmegebühr bis spätestens Montag bei einem der Tutoren zu
zahlen.

Hier ist noch eine kleine Packhilfe. Das brauchst du unbedingt: 
\begin{itemize}\itemsep 0pt
    \item Bettlaken, Kopfkissenbezug und entweder einen Bettbezug oder
          Schlafsack
    \item Mindestens einen (gerne mehr!) Holzscheit(e) für ein großes
          Lagerfeuer, Taschenlampe
    \item deinen Lieblingsbrotbelag (Süßes gibt es!)
    \item was du sonst für eine Übernachtung brauchst 
\end{itemize}

Das kann mit: 
\begin{itemize}\itemsep 0pt
    \item Brett- und Kartenspiele
    \item Bälle, Frisbees, Tischtennisschläger, \dots
    \item Musikinstrumente
\end{itemize}

% TODO: Issue 4

Wir sehen uns Montag bei der Vorbesprechung! Dort kannst du alle deine Fragen
stellen und dich natürlich auch noch anmelden.

\end{multicols}
\begin{center}
\vfill
\includegraphics[scale=.8]{comics/435}
\end{center}
\begin{multicols}{2}
