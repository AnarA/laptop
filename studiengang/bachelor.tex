\begin{multicols}{2}[\subsection{Der Bachelor-Studiengang}]

Seit dem Wintersemester 2006/07 sind die Studiengänge Mathematik und
Wirtschaftsmathematik an der Universität Hamburg, sowie die
Lehramtsstudiengänge auf das Bachelorsystem umgestellt.  Bevor wir euch auf den
nächsten Seiten die Studiengänge näher vorstellen wollen, werden wir euch vorab
ein paar grundlegende Informationen geben, die alle Studiengänge betreffen.
Alle Fragen, die sich vielleicht noch beim Lesen dieser Seiten ergeben, könnt
ihr am besten während der Studienberatung stellen, aber auch darüber hinaus
sind die Tutoren für alle Fragen rund um das neue Bachelorstudium ansprechbar.
Im Internet findet ihr die aktuellsten Informationen unter
\url{http://www.math.uni-hamburg.de/teaching2/index.html}

\subsubsection{Der grobe Überblick}

Das Studium ist auf 6 Semester ausgelegt. Anschließend kann man sich für einen
der weiterführenden Master-Studiengänge, ausgelegt auf vier Semester,
entscheiden oder auch direkt in das Berufsleben starten. Für Lehramtsstudenten
ist dabei ein Master nötig um an Schulen unterrichten zu dürfen.

Das Bachelorstudium ist in \emph{Module} organisiert. Ein Modul kann mehrere
Veranstaltungen, \emph{Vorlesungen}, \emph{Seminare}, \emph{Übungen}, umfassen,
die inhaltlich zusammenhängen.  Jedes Modul wird mit einer \emph{Prüfung}
abgeschlossen.  Die Ergebnisse der meisten Modulprüfungen gehen in die
Abschlussnote ein. In manchen Fällen werden Teilmodulprüfungen abgenommen.
Details über die einzelnen Module findet ihr in den \emph{Fachspezifischen
Bestimmungen} (FSB) der jeweiligen Fächer.  Es gibt \emph{Pflichtmodule},
\emph{Wahlpflichtmodule}, \emph{Ergänzungsmodule} und frei wählbare
\emph{Wahlmodule}.  Dabei kann ein Modul auch über zwei Semester gehen wie zum
Beispiel die Analysis- oder die Lineare Algebra \& Analytische
Geometrie-Vorlesung. 

In den ersten Semestern habt ihr bei der Belegung im mathematischen Bereich
nicht sehr viele Möglichkeiten, das Studium nach euren Vorstellungen zu
gestalten. Das wird in der zweiten Studienphase mit der Wahl der
Wahlpflichtmodule einfacher.  Im Vergleich mit anderen mathematischen
Fachbereichen zeichnet sich dabei der Fachbereich Mathematik an der Universität
Hamburg durch eine besonders große Vielfalt aus, die sich in der fachlichen
Breite der Schwerpunkte widerspiegelt. Somit habt ihr bei der Wahl eurer
Wahlpflichtmodule vielzählige Angebote

Eure Module im Bereich Mathematik sind meist mit einer Vorlesung und einer
zugehörigen Übung aufgebaut.  In den Übungsstunden werden die Übungsaufgaben zu
den Vorlesungen verteilt und besprochen.  Die Übungen sind wichtig, da ihr euch
zum Lösen der Aufgaben mit dem späteren Prüfungsstoff beschäftigen müsst, aber
auch, weil das erfolgreiche Lösen der Übungsaufgaben in der Regel eine
Zulassungsvoraussetzung für die Modulabschlussprüfung ist.  Des Weiteren ist
ein wichtiger Bestandteil der Übungen das Vortragen der Hausaufgaben vor der
Gruppe an der Tafel.  Auch dieses kann eine Zulassungsvoraussetzung sein. Die
genauen Bestimmungen wird der jeweilige Veranstalter zu Beginn des Semesters
bekannt geben.  Zusätzlich wird in der Analysis und der Linearen Algebra \&
Analytischen Geometrie noch ein freiwilliges \emph{Tutorium} angeboten, in
welchem die Inhalte der Vorlesung wiederholt und vertieft und Fragen
beantwortet werden.

% TODO: Erklärung von SWS prüfen.

Alle Module sind mit einer bestimmten Anzahl an \emph{Leistungspunkten} (LP)
bewertet. Diese geben einen Eindruck vom Arbeitsaufwand, der zum erfolgreichen
Absolvieren des Moduls notwendig ist. Ein Richtwert hierfür sind etwa 30
Arbeitsstunden pro Leistungspunkt.  Vorgesehen sind etwa 30 LP pro Semester und
180 LP bis zur Beendigung des Bachelorstudiums.  Es ist normal, dass man in dem
einen Semester ein paar LP mehr und im anderen Semester ein paar LP weniger
macht, es ist aber eine Gleichverteilung der Punkte sinnvoll.  Der Zeitumfang
einer Veranstaltung wird in \emph{Semesterwochenstunden} (SWS) ausgedrückt.
Diese Angabe bezieht sich auf die Stunden, die durch die Veranstaltung in der
Universität innerhalb einer Woche in Anspruch genommen werden.

Im weiteren Verlauf eures Studiums werdet ihr auch Seminare belegen.  In
Seminaren erarbeitet man vorgegebene Themen anhand von Literatur und hält vor
den Seminarteilnehmern einen Vortrag über sein Thema.  Um dieses zu üben, gibt
es \emph{Proseminare}, die euch einen guten Einstieg in diese Arbeitsweise
geben. Im Studium Bachelor Mathematik und Bachelor Wirtschaftsmathematik ist
ein Proseminar verpflichtend vorgesehen. 
\end{multicols}

\clearpage
