\subsection{Grundlagen der Mathematik (M1) f\"ur Studierende der Lehr\"amter
Primar- und Sekundarstufe I sowie Sonderschulen} 

Mein Studium begann ich 1979 an der TU M\"unchen, und schloss 1986 mit dem
ersten Staatsexamen f\"ur das Lehramt an Gymnasien in den F\"achern Mathematik
und Physik ab. Nach Stationen in M\"unchen, Tucson (Arizona), und wieder
M\"unchen, kam ich 1999 nach Hamburg.

Die solide mathematische Ausbildung von Studierenden des Lehramts liegt mir
besonders am Herzen.  Sie werden sehen, dass Mathematik Spa\ss{} machen kann.
Wenn Sie eine zun\"achst unl\"osbar scheinende Aufgabe schlie"slich doch
herausbekommen, k\"onnen Sie ein tiefe Befriedigung erleben.  Freilich setzt
das ausdauerndes, kontinuierliches Arbeiten und fundierte Kenntnisse der
Schulmathemetik voraus. Die Sch\"onheit der Mathematik erschlie"st sich nur den
Personsn, die sich intensiv darum bem\"uhen.

Inhaltlich wird es im ersten Semester um Grundlagen gehen, wie etwa
\textit{Mengen, Logik, Relationen, Funktionen, vollst\"andige Induktion, Aufbau
des Zahlensystems,} uvam.

Ich freue mich darauf Ihnen in den kommenden Semestern die Sch\"onheit der
Mathematik nahebringen zu d\"urfen und hoffe auf eine gute Zusammenarbeit.  Bis
dahin w\"unsche ich Ihnen eine interessante und abwechslungsreiche OE. Nutzen
Sie die Zeit, Kommilitonen auch anderer Studienrichtungen kennen zu lernen.

Viel Erfolg in Ihrem ersten Semester!

Hubert Kiechle\\
Bereich Geometrie\\
Fachbereich Mathematik\\
Telefon 42838\,5186, Raum 224\\
hubert.kiechle [at] uni-hamburg.de
