Das Studien-Infonetz der Universität Hamburg, STiNE, ist ein internetbasiertes
Informations- und Kommunikationssystem für Studium und Lehre. Dieses System
wurde eingeführt, um den Uni-Alltag leichter zu machen. Mithilfe dieses
Online-Portals kann der Student an der Universität Hamburg seine Vorlesungen
verwalten, sich für Prüfungen an- und abmelden und seine Noten in einer
übersichtlichen Form abrufen.

Derzeit werden die an der Universität Hamburg angebotenen Studiengänge im
Rahmen des \emph{Bologna-Prozesses} auf das Bachelor- und Masterstudiensystem
umgestellt. Im Zuge der Umstellung werden in den neuen Studiengängen Module
eingeführt, die mit einem internationalen Leistungspunktesystem bewertet
werden. Das Bachelorstudium wird nicht durch eine große Abschlussprüfung
beendet, sondern jedes Modul schließt mit einer Prüfung ab, die automatisch zu
einem Teil in die Note des Bachelors eingeht. Mit STiNE kann man sich über die
jeweiligen Prüfungsordnungen in den Bachelorstudiengängen informieren.
Gleichzeitig gibt es die Möglichkeit, den aktuellen Stand seines
Leistungspunktekontos abzufragen und sich über Ergebnisse seiner absolvierten
Prüfungen zu informieren.

Die Bestimmungen zum Datenschutz bei STiNE sind sehr streng. Keiner der
Dozenten kann die alten Prüfungsergebnisse seiner Studenten einsehen. Nur dem
Student selber und dem Prüfungsamt ist dieses erlaubt. Eine Note kann nur mit
erheblichem Aufwand geändert werden.

Bei der Anmeldung zu einer Vorlesung muss man sich in der Regel gleich zur
Prüfung mit anmelden. Hierzu benötigt man jeweils einen iTAN aus seiner
iTAN-Liste. Diese sollte spätestens erneuert werden, wenn nur noch zwei iTANs
übrig sind, da man eine für die Beantragung einer neuen Liste und eine für die
Freischaltung ebendieser benötigt.  Innerhalb der Anmeldefrist können
Anmeldungen jederzeit storniert werden. Auch nach der Zuteilung eines
Teilnehmerplatzes ist eine Stornierung noch innerhalb von zwei Wochen nach
Ablauf der Anmeldefrist möglich.

Bei der Eingabe des automatisch generiertem Passworts sollte man beachten, dass
bei dem Bogen für die Anmeldedaten eine Standardschrift verwendet wird, die es
unter Umständen erschwert das Kennwort eindeutig zu identifizieren. So ähneln
sich die Null und das große O sowie die Eins, das kleine L und das große i. Um
die Sicherheit seiner Daten zu bewahren sollte man das Passwort gleich nach dem
ersten Anmelden ändern. Hierbei solltest du ein ausreichend sicheres Passwort
verwenden.

Eine weitere nützliche Option bei STiNE ist der Terminexport, der es einem
ermöglicht die Termine einer Woche in einer Datei zu speichern und diese in
einen elektronischen Terminkalender zu importieren. Von dort aus können die
Termine mit dem Handy synchronisieren werden. Mit dieser technischen Spielerei
hat man ganz leicht seine Vorlesungen und Veranstaltungen im Überblick.

\begin{center}
\includegraphics[scale=0.5725]{comics/936}
\end{center}
