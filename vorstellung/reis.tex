\subsection{Analysis}

Ihr Mathematikstudium beginnt mit den Veranstaltungszyklen ``Lineare Algebra''
und ``Analysis''. Eine solche Aufteilung wird seit vielen Jahrzehnten weltweit
an fast allen Universit\"aten praktiziert und hat sich auch bew\"ahrt.  Diese
Vorlesungen bilden die Grundlage f\"ur s\"amtliche weiteren Veranstaltungen
Ihres Studiums.

W\"ahrend bei der linearen Algebra neben grundlegenden algebraischen Strukturen
das lineare Gleichungssystem im Vordergrund steht, geht es in meiner
Veranstaltung ``Analysis`` eher um sogenannte ``topologische Aspekte''. D.h.,
es geht um Fragestellungen, die damit zusammenh\"angen, ob Zahlen (oder auch
Vektoren o.\"A.) ``nahe beieinanderliegen``. Die daraus herleitbaren Begriffe
wie Konvergenz, Grenzwert, Stetigkeit, Differenzierbarkeit und Integrale sind
gleichwohl zentrale Konzepte der Analysis und werden in dieser Veranstaltung
behandelt. Diese Begriffe kommen Ihnen sicherlich bekannt vor; wir werden sie
jedoch von Grund auf neu einf\"uhren.  Anders als vielleicht in der Schule
stehen Universit\"atsvorlesungen eher axiomatische Zug\"ange im Vordergrund.
Das macht Anf\"angern erfahrungsgem\"a\ss\ h\"aufig Schwierigkeiten. Diese
k\"onnen Sie \"uberwinden, indem Sie beharrlich ``am Ball bleiben'': Dazu
geh\"ort etwa das Bearbeiten von \"Ubungsbl\"attern und die regelm\"a\ss ige
Nachbearbeitung des Vorlesungsstoffs. Bei Letzterem werden wir Sie in Form der
Anbietung sogenannter ``Tutorien'' unterst\"utzen; dort k\"onnen Sie
beispielsweise Fragen zum Vorlesungsinhalt und darüber hinaus stellen, welche
meine Mitarbeiter und ich Ihnen gerne beantworten wollen.

Das die mathematische Welt strikt und disjunkt in Analysis und lineare Algebra
aufgeteilt ist, wird bereits in Ihrem zweiten Semester widerlegt. Dort werden
wir, wenn wir in ``Analysis II`` Funktionen mit mehreren Ver\"anderlichen
betrachte, auch auf Konzepte der linearen Algebra zugreifen, wie etwa Matrizen.
In Ihrem weiteren Studium werden Sie sehen, dass sich die Inhalte beider
Veranstaltungen immer mehr ineinander verzahnen. Mein Interesse besteht darin,
dass sie m\"oglichst viel aus der Analysis lernen und dass Sie im Rahmen meiner
Vorlesung ebenfalls einen gewissen Gesamteindruck der Mathematik bekommen, um
Ihnen somit eine gute Grundlage f\"ur ein erfolgreiches Mathematikstudium zu
erm\"oglichen.

Ich selbst arbeite in meiner mathematischen Forschung an Steuerungs-,
Regelungs- und Approximationsproblemen bei Differentialgleichungen und
besch\"aftige mich insbesondere mit Anwendungen in der Elektrotechnik. Im Jahr
1998 begann ich in Kaiserslautern mit dem Studium der Mathematik und ging nach
meiner Promotion 2006 als wissenschaftlicher Mitarbeiter an die Technische
Universit\"at Berlin. Nach einem jeweils etwa einj\"ahrigen Engagement als
Gastwissenschaftler an der Rice University in Houston/Texas und als
Vertretungsprofessor an der Technischen Universit\"at Hamburg-Harburg nahm ich
im vergangenen Jahr 2011 einen Ruf auf eine Professur an der Universit\"at
Hamburg an.

Ich freue mich, dass man meinem Wunsch nach dem Halten einer
Anf\"angervorlesung f\"ur Mathematikstudenten entsprochen hat und hoffe auf
angenehme und erfolgreiche Zusammenarbeit mit Ihnen.

Ich w\"unsche Ihnen einen angenehmen Einstieg ins Studium und viel Erfolg!
Timo Reis


Prof.~Dr.~Timo Reis
Fachbereich Mathematik
Universit\"at Hamburg
Tel. 42838-5111, Raum 123
Mail: timo.reis@math.uni-hamburg.de

