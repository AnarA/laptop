\begin{multicols}{2}[\subsection{Alle wichtigen Anlaufstationen und Adressen}]

Wir zeigen euch hier (ohne Garantie auf Vollständigkeit) eine Liste der
wichtigsten Ansprechpartner und Institutionen. Am besten schaut ihr aber selber
im Internet nach und sucht euch den aktuellen Verantwortlichen heraus. Die
wichtigsten Internetadressen dafür sind:

\begin{itemize}\itemsep 0pt
    \item Fachbereichs Mathematik:\\
          \url{http://www.math.uni-hamburg.de}
    \item Studienbüro des Fachbereichs Mathematik:\\
          \url{http://www.math.uni-hamburg.de/studienbuero/}
    \item Beauftragte am Fachbereichs Mathematik:\\
          \url{http://www.math.uni-hamburg.de/contact/beauftragte.html}
    \item Alle Mitarbeiter des Fachbereichs Mathematik:\\
          \url{http://www.math.uni-hamburg.de/contact/mitarbeiter.html}
    \item Fachbereich Wirtschaftswissenschaften:\\
          \url{http://www.wiso.uni-hamburg.de}
    \item TU Harburg:\\
          \url{http://www.tuhh.de}
    \item Zentrum für Studierende:\\
          \url{http://www.verwaltung.uni-hamburg.de/vp-1/3/33/index.html}
    \item Zentrum für Studienberatung und Psychologische Beratung:\\
          \url{http://www.verwaltung.uni-hamburg.de/vp-1/3/34/index.html}
\end{itemize}

Für die verschiedenen Studiengänge sind Mailinglisten eingerichtet. Alle
relevanten Informationen werden per E-Mail an alle eingetragenen Adressen
verschickt. Jeder sollte sich daher unbedingt in seine Mailingliste eintragen.
Dieses könnt ihr auf der entsprechenden, folgenden Seite machen.

\begin{center}
\includegraphics[scale=.65]{comics/733}
\end{center}

\begin{itemize}\itemsep 0pt
    \item Bachelor Mathematik\\
          \url{https://mailman.rrz.uni-hamburg.de/mailman/listinfo/bama}
    \item Bachelor Wirtschaftsmathematik\\
          \url{https://mailman.rrz.uni-hamburg.de/mailman/listinfo/bawima}
    \item Lehramt an Gymnasien und Berufliche Schulen\\
          \url{https://mailman.rrz.uni-hamburg.de/mailman/listinfo/lamagymbs}
    \item Lehramt an der Primar- und Sekundarstufe I, Lehramt an
          Sonderschulen\\
          \url{https://mailman.rrz.uni-hamburg.de/mailman/listinfo/lamapss}
\end{itemize}

Wenn ihr euch mit Professoren oder anderen Mitarbeitern treffen wollt, ist es
häufig sinnvoll, den Kontakt zunächst per E-Mail herzustellen und dann einen
Termin abzusprechen. Wegen kleinerer Angelegenheiten kann man die meisten
Professoren bei deren Anwesenheit häufig auch außerhalb der Sprechstunden
aufsuchen.
\end{multicols}

\subsubsection{Studienfachberatung}

%% Informationen sind online unter
%% http://www.math.uni-hamburg.de/teaching/service/studienfachberatung.html
%% zu finden.

\begin{tabularx}{\textwidth}{|X|X|X|X|}
\hline Bachelor Mathematik&Studienbüro Mathematik&Di, Mi, Do \hfill 10-12
    &Räume E 14, E16\\
    &studienbuero@math.uni-hamburg.de&Di, Do \hfill13-15&\\
\hline Bachelor Wirtschaftsmathematik&Prof. Dr. Hans Daduna&Di \hfill 10-11
    &Raum T17 \\
&hans.daduna@math.uni-hamburg.de&&+49 40 42838-4930\\
&&&\\
&Prof. Dr. Natalie Neumeyer&Di \hfill 14-15&Raum T13\\
&natalie.neumeyer@math.uni-hamburg.de&&+49 40 42838-4907\\
\hline Bachelor Lehramtsstudiengänge&&&\\
Lehramt Gymnasium und Berufliche Schulen&Prof. Dr. Ingo Runkel&&Raum 309\\
    &ingo.runkel@math.uni-hamburg.de&&+49 40 42838-5172\\
&&&\\
Lehramt Primar- und Sekundarstufe I, Lehramt an Sonderschulen
    &Prof. Dr. Andrea Blunck&Di \hfill 14-15&Raum 211\\
    &andrea.blunck@math.uni-hamburg.de&&+49 40 42838-5160\\
&&&\\
\hline Master Mathematik
    &\multicolumn{3}{|l|}{Studienbüro Mathematik, siehe Bachelor Mathematik}\\
\hline
\end{tabularx}

\subsubsection{Spezielle Fragen zu den Prüfungen für das Lehramt}

\begin{tabularx}{\textwidth}{|X|X|X|X|}
\hline Lehrerprüfungsamt&Mümmelmannsberg 75&Mo, Do \hfill 9-12
    &+49 04 42854-7611\\
    &&Di \hfill 14-15&\\
\hline
\end{tabularx}

\subsubsection{BAföG-Vertrauensdozenten}

\begin{tabularx}{\textwidth}{|X|X|X|}
\hline Prof. Dr. Hans Joachim Oberle&Di,Fr \hfill 9-10&Raum 120\\
       oberle@math.uni-hamburg.de&&+49 40 42838-5113\\
\hline Prof. Dr. Thomas Andreae&Di, Fr \hfill 14-15&Raum 238\\
       andreae@math.uni-hamburg.de&&+49 40 42838-5196\\
\hline
\end{tabularx}

\subsubsection{Beauftragter für ausländische Studierende}

\begin{tabularx}{\textwidth}{|X|X|X|}
\hline Prof. Dr. Ingenuin Gasser&Di \hfill 11-12&Raum 105\\
       ingenuin.gasser@math.uni-hamburg.de&&+49 40 42838-5128\\
\hline
\end{tabularx}

\subsubsection{Fachschaftsrat Mathematik}

\begin{tabularx}{\textwidth}{|X|X|X|}
\hline Fachschaftsrat Mathematik&&Raum T30\\
       fsr@math.uni-hamburg.de&&\\
\hline
\end{tabularx}

\subsubsection{Wichtige Adressen und Telefonnummern}

\begin{tabularx}{\textwidth}{|X|X|X|X|}
\hline FSR Wirtschaftswissenschaften&Von-Melle-Park 5&Raum 0070&+49 40 441266\\
       \url{http://wiwifsr.de/}&&&\\
\hline FSR Physik&Jungiusstraße 9&Raum 020&+49 40 352 202\\
       \url{http://fsrix.physnet.uni-hamburg.de/}&&&\\
\hline FSR Informatik&Vogt-Kölln-Str. 30&C-215&+49 40 5404228\\
\url{http://www.informatik.uni-hamburg.de/Fachschaft/wiki/index.php/Fachschaftsrat}&&&\\
\hline FSR Elektrotechnik&Schwarzenbergstraße 95&Raum E 0.098&+49 40 42878-2975\\
       \url{http://fsr-etit.de/}&&&\\
\hline FSR Maschinenbau&Schwarzenbergstraße 95&Raum 0.101&+49 40 42878-4008\\
       \url{http://cgi.tu-harburg.de/~fsrmwww/doku.php?id=start}&&&\\
\hline Studierendenwerk Hamburg&Von-Melle-Park 2&&+49 40 41902-0\\
       \url{http://www.studierendenwerk-hamburg.de}&&&\\
\hline
\end{tabularx}
\clearpage
