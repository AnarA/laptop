\begin{multicols}{2}[\section{Bachelor Mathematik}]

In den ersten drei Semestern, die zur \emph{ersten Studienphase} gehören,
werden euch in Pflichtmodulen die mathematischen Grundlagen in den Bereichen
Analysis, Lineare Algebra, Numerische Mathematik und Mathematische Stochastik
vermittelt. Zusätzlich müsst ihr die Module \emphm{Programmiermethoden} und
\emphm{Softwarepraktikum} zum Erlernen \emphm{Allgemeiner Berufsqualifizierende
Kompetenzen} (ABK) belegen. Abgeschlossen wird die erste Studienphase durch ein
Proseminar, das für das vierte Semester vorgesehen ist.  In der \emph{zweiten
Studienphase}, viertes bis sechstes Semester, könnt ihr den mathematischen Teil
eures Studiums aus einer Vielzahl von Wahlpflichtmodulen nach euren
persönlichen Neigungen selber gestalten. 

In diesem Wintersemester hört ihr neben zwei Veranstaltungen aus der reinen
Mathematik, \emphm{Analysis I} und \emphm{Lineare Algebra \& Analytische
Geometrie I}, Vorlesungen in eurem \emph{Ergänzungsfach}. Im 2. Semester hört
ihr \emphm{Analysis II}, \emphm{Lineare Algebra \& Analytische Geometrie II}
sowie eventuell eine weitere Veranstaltung eures Ergänzungsfaches oder besucht
eine kleinere Veranstaltung aus der Mathematik. In den letzten Wochen vor
Beginn des dritten Semesters müsst ihr noch einen Programmierkurs absolvieren,
der euch auf die Veranstaltung \emphm{Numerische Mathematik} vorbereitet. Das
3.  Semester setzt sich dann aus den Veranstaltungen \emphm{Höhere Analysis},
\emphm{Mathematische Stochastik} und \emphm{Numerische Mathematik} sowie einem
\emphm{Softwarepraktikum} zusammen. 

Der Umfang der einzelnen Veranstaltungen sieht wie folgt aus: Alle oben
genannten Mathematik-Module bestehen aus 4 SWS Vorlesung und 2 SWS Übung.
Begleitend zu den Modulen \emphm{Analysis} und \emphm{Lineare Algebra \&
Analytische Geometrie} wird ein freiwilliges 2 SWS Tutorium angeboten. Der
Programmierkurs findet als zweiwöchiger Blockkurs am Ende der vorlesungsfreien
Zeit statt. Die Inhalte des Softwarepraktikums sollt ihr euch selbstständig
erarbeiten, zeitlich neben den Mathematik-Vorlesungen. Im Softwarepraktikum
gibt es keine Vorlesung oder Übung, sondern nur eine Sprechstunde bei
Problemen.

In der zweiten Studienphase habt ihr die Möglichkeit, euch euren Interessen
entsprechend auszurichten. Dabei müsst ihr aus einem großen Kanon an
Wahlpflichtveranstaltungen des Bachelors Module im Umfang von insgesamt 36 LP
sowie im vieten Semester ein Proseminar und im fünften Semester ein Seminar
belegen. Bei der Auswahl der Module sollte auf einen \emph{sinnvollen
Studienaufbau} und eine \emph{hinreichende Breite} geachtet werden.  Im dritten
Semester findet eine verpflichtende \emph{Studienfachberatung} statt, die euch
darüber informieren soll, was genau sich hinter diesen Begriffen verbirgt.
Während eures Studiums solltet ihr ein Berufspraktikum absolvieren. Alternativ
könnt ihr dieses ABK-Modul jedoch auch durch das Leiten einer Übungsgruppe oder
durch die Arbeit an einem Projekt abdecken.  Die zweite Studienphase wird durch
die Bachelorarbeit abgeschlossen. Diese baut in der Regel auf einem
Wahlpflichtmodul oder dem Seminar auf.

Zusätzlich zu den Mathematik-Vorlesungen müsst ihr noch Veranstaltungen in
einem Ergänzungsfach belegen, in dem ihr die Grundlagen eines wichtigen
Anwendungsgebiets der Mathematik kennenlernt. Dieses ist ein Fach mit einem
Bezug zur Mathematik, das ab dem ersten Semester gehört werden sollte. Auf
Seite \pageref{page:erg} dieses Laptops wirst du umfassend über die
Ergänzungsfächer informiert.

Das Modulspektrum des Bachelor wird durch den \emph{Wahlbereich}
vervollständigt. Prinzipiell stehen euch dafür alle an der Universität Hamburg
angebotenen Veranstaltungen zur Verfügung.  In eurem eigenen Interesse solltet
ihr jedoch auf eine sinnvolle Verteilung der 21 LP Wert legen. So ist es nicht
nur wenn ihr ein Masterstudium an den Bachelor anschließen wollt vernünftig,
einen Teil der Wahlmodule mit Mathematikveranstaltungen oder zusätzlichen
Veranstaltungen im Ergänzungsfach abzudecken.  Doch auch das Belegen von
Sprach- oder Computerkursen kann zur zusätzlichen Qualifikation für euren
späteren Traumjob sinnvoll sein.
\end{multicols}
\clearpage

\subsection{Musterstudienplan für den Studiengang B.Sc. Mathematik}

\begin{center}
\begin{tabular}{||l||l|l|l|l|l|l||}
\hhline{|t:=:t:=t=t=t=t=t=:t|}
\hspace*{25mm} & 1. Semester\hspace*{8ex} & 2. Semester\hspace*{8ex} & 3. Semester\hspace*{8ex} & 
4. Semester\hspace*{8ex} & 5. Semester\hspace*{8ex} & 6. Semester\hspace*{8ex} \\
\hhline{|:=::======:|}
Pflichtbereich & Analysis I& Analysis II&Höhere Analysis& Proseminar &Seminar& Bachelorarbeit\\
\hhline{||~||~|~|~|~|~|~||} &(9 LP)&(9 LP)&(9 LP)&(4 LP) &(6 LP)& (12 LP)\\
\hhline{||~||~|~|~|~|~|~||} &&&&&&\\
\hhline{||~||~|~|~|~|~|~||} &Lineare Algebra & Lineare Algebra & Mathematische & & &\\ 
\hhline{||~||~|~|~|~|~|~||} &und Analytische & und Analytische & Stochastik & & &\\ 
\hhline{||~||~|~|~|~|~|~||} &Geometrie I&Geometrie II & (9 LP) & &&\\ 
\hhline{||~||~|~|~|~|~|~||} &(9 LP)&(9 LP)& &&&\\
\hhline{||~||~|~|~|~|~|~||} &&&Numerische&&&\\
\hhline{||~||~|~|~|~|~|~||} &&&Mathematik&&&\\
\hhline{||~||~|~|~|~|~|~||} &&&(9 LP)&&&\\ 
\hhline{||~||~|~|~|~|~|~||} &&&&&&\\
\hhline{|:=::======:|}
\hhline{||~||~|~|~|~|~|~||} Wahlpflicht-&&&&Vertiefungs-&Vertiefungs-&Vertiefungs- \\
\hhline{||~||~|~|~|~|~|~||} bereich&&&&module&module&module \\
\hhline{||~||~|~|~|~|~|~||} &&&&(18 LP)&(9 LP)&(9 LP) \\
\hhline{||~||~|~|~|~|~|~||} &&&&&&\\
\hhline{|:=::======:|}
ABK-Bereich&& Programmier-&Softwarepraktikum&&Betriebspraktikum/& \\
\hhline{||~||~|~|~|~|~|~||} &&methoden&(4 LP)&&Projekt/Tutorium&\\
\hhline{||~||~|~|~|~|~|~||} &&(5 LP)&&&(5 LP)&\\
\hhline{||~||~|~|~|~|~|~||} &&&&&&\\
\hhline{|:=::======:|}
Ergänzungs-&Ergänzungs- &Ergänzungs-&&&Ergänzungs-&Ergänzungs-\\
\hhline{||~||~|~|~|~|~|~||} fachbereich&fachmodule&fachmodule&&&fachmodule&fachmodule\\
\hhline{||~||~|~|~|~|~|~||} &(7 LP)&(3 LP)&&&(5 LP)&(9 LP)\\
\hhline{||~||~|~|~|~|~|~||} &&&&&& \\
\hhline{|:=::======:|}
Wahlbereich&Wahlmodul&Wahlmodul&&Wahlmodul&Wahlmodul&\\
\hhline{||~||~|~|~|~|~|~||} &(4 LP)&(4 LP)&&(8 LP)&(5 LP)&\\
\hhline{||~||~|~|~|~|~|~||} &&&&&& \\
\hhline{|b:=:b:======:b|}
\end{tabular}
\end{center}

\clearpage
