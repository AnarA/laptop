\begin{tabularx}{\textwidth}{|l|X|X|X|}
% I. Bachelor (Mathematik, Wirtschaftsmathematik, Mathematik Lehramt an Gymnasien und Lehramt an Beruflichen Schulen)
\hline 65--001 	& Orientierungseinheit f"ur Studienanf"anger/innen (Mathematik und Wirtschaftsmathematik Bachelor, alle Lehr"amter mit Mathematik als Unterrichtsfach)
			& Block\-ver\-an\-stal\-tung vom 04.-13.10.12; 4 SWS, ganzt"agig, Beginn: 04.10.12, 09:30 Uhr Geom H1
			& Dr. Hubert Kiechle, Martin Rybicki \\
\hline 65--002 	& Tutorium f"ur ausl"andische Studierende (insbesondere im Rahmen von ERASMUS)
			& 2 SWS, Zeitpunkt noch unbekannt
			& Prof. Dr. Armin Iske \\
\hline 65--003 	& Vorkurs Mathematik (Blockveranstaltung vom 24.09 - 02.10.12)
			& 2,83 SWS, Mo.-Do. 09:30-11:00, 11:30-13:00, 17:00-18:15, Fr. 09:30-11:00, 11:30-13:00, 16:15-17:30, Mo. 01.10.-Di. 02.10. 09:30-11:00, 11:30-13:00, 16:15-17:30, Hörsaal C Chemie, Beginn 24.09.2012
			& Dr. Anna Posingies \\
\hline 65--004 	& "Ubungen zu Vorkurs Mathematik (Blockveranstaltung vom 24.09.-02.10.12) (7 Gruppen)
			& 1 SWS, Mo.-Do. 15:15-16:45, Fr. 14:15-15:45, Mo. 01.10.-Di. 02.10. 14:15-15:45, Geom 241, 430, 431, 432, 434, 435, 1240
			& N.N. \\
\hhline{|=|=|=|=|} 65--011 & Lineare Algebra und Analytische Geometrie I
			& 4 SWS, Mo. 14:15-15:45, Mi. 08:15-09:45, Geom H1, Beginn: 15.10.12
			& Prof. Dr. Birgit Richter \\
\hline 65--012 	& "Ubungen zu Lineare Algebra und Analytische Geometrie I (10 Gruppen)
			& 2 SWS, Di. 14:15-15:45, Geom 241, 435, Di. 16:15-17:45, Geom 435, 1240, Mi. 10:15-11:45, Geom 434, (431, 435 f"ur LG/LBS), Mi. 12:15-13:45, Geom 431, 434, 435, Beginn: 16.-17.10.12
			& Prof. Dr. Birgit Richter, N.N. \\
\hline 65--014 	& Tutorium zu Lineare Algebra und Analytische Geometrie I (9 Gruppen)
			& 1 SWS, Mo. 12:15-13:00, Geom 434, 435 (\& 431 f"ur LG/LBS), Mo. 13:15-14:00, Geom 435 (\& 431 f"ur LG/LBS), Di. 12:15-13:00, Geom 241, 430, Di. 13:15-14:00, Geom 241, 430, Beginn: 15.-16.10.12
			& N.N. \\
\hline 65--021 	& Analysis I
			& 4 SWS, Mi. 14:15-15:45, Fr. 14:15-15:45 Geom H1, Beginn: 17.10.12
			& Prof. Dr. Timo Reis \\
\hline 65--022 	& "Ubungen zu Analysis I (9 Gruppen)
			& 2 SWS, Do. 08:15-09:45, Geom 241, 1240, Do. 10:15-11:45, Geom 241, 431, 1240, Fr. 08:15-09:45, Geom 241, 431, Fr. 16:15-17:45, Geom 431, 435 (LG/LBS), Beginn: 17.-19.10.12
			& Prof. Dr. Timo Reis, N.N. \\
\hline 65--024 	& Tutorium zu Analysis I (5 Gruppen)
			& 2 SWS, Do. 08:15-09:45, Geom 431, (435 LG/LBS), Do. 12:15-13:45, Geom 431, (435 LG/LBS), Do. 14:15-15:45, Geom 431, Beginn: 18.10.12
			& N.N. \\
\hline
\end{tabularx}
\begin{tabularx}{\textwidth}{|l|X|X|X|}
%Keine Markierung, welche LG/LBS sind.
% II. Bachelor (Mathematik Lehramt der Primarstufe und Sekundarstufe I sowie Lehramt an Sonderschulen)
\hline 65--301 	& Grundlagen der Mathematik
			& 4 SWS, Mo. 14:15-15:45, Mi. 08:15-09:45, Geom H2, Beginn: 15.10.10
			& Dr. Hubert Kiechle \\
\hline 65--302 & "Ubungen zu Grundlagen der Mathematik (5 Gruppen)
			& 2 SWS, Mo. 16:15-17:45, Geom 241, 431, 435, 1240, Di. 14:15-15:45, Geom 1240, Beginn: 15.-16.10.12
			& Dr. Hubert Kiechle, N.N. \\
% Enthalten oben.
%\hline 65-252 & "Ubungen zu Grundlagen der Mathematik f"ur Studierende der Lehr"amter Primarstufe und Sekundarstufe I sowie Sonderschulen (5 Gruppen)
%              & 2st., Mo 16.15-17.45 Geom 431,434,435,1241,Di 14.15-15.45 Geom 1241 Beginn: 19.10.10
%              & Hans-Joachim Samaga\\
% Ist aktuell. (Stand 18.9.2010)
\hline 65--324 & Arbeitsgruppenbetreuung zu Grundlagen der Mathematik, Grundbildung Analysis und Grundbildung Stochastik (5 Gruppen)
			& 4 SWS, Do. 18:00-21:00, Geom 430, 431, 432, 434, 435, Beginn: 18.10.12
			& Dr. Hans-Christian von Bothmer, Dr. Hubert Kiechle, Dr. Susanne Margret Koch \\
%entf"allt.
%\hhline{|=|=|=|=|} 65--902	& Geschichte der Mathematik
%			& 2 St., Fr. 12.15 - 13.45, Geom H1, Beginn: 22.10.10
%			& Prof. Dr. Thomas Sonar \\
%
%\hline 65--922	 & Seminar zur Vorlesung 65--902: Geschichte der Mathematik
%			& 2 St., Fr. 14:15 - 15:45, Geom E11/13, Beginn: 22.10.10
%			& Prof. Dr. Thomas Sonar \\

% III. Master (Mathematik Lehramt der Primarstufe und Sekundarstufe I sowie Lehramt an Sonderschulen)%\hline 65--303 & Frauen in der Geschichte der Mathematik
%			& 2 St., Do. 14:45 - 15:45, Geom H4, Beginn: 21.10.10
%			& Prof. Dr. Andrea Blunck \\
%
%\hline 65--304 & "Ubungen zu Frauen in der Geschichte der Mathematik (2 Gruppen)
%			& 1 St., Do. 12:15 - 13:45, Geom H3, Beginn: 28.10.10
%			& Prof. Dr. Andrea Blunck \\
% IV, V, VI, VII fehlen! (TODO)
% Nebenfach: Physik
\hhline{|=|=|=|=|} 66--100 	& Physik I
			& 4 SWS, Di. 14:00-15:30, Do. 09:00-10:30, Jungius 9, H"ors II, Beginn: 23.10.12
			& Prof. Dr. Ulrich Merkt \\
\hline 66--101 	& Einf"uhrung in die Theoretische Physik I
			& 3 SWS, Di. 15.45-16.45, Do. 10.45-12.00, Jungius 9, H"ors II, Beginn: 23.10.12
			& Prof. Dr. Ludwig Mathey \\
\hline 66--102 	& "Ubungen zur Physik I und Einf"uhrung in die Theoretische Physik I
			& 3 SWS, Do. 13.00-15.15, 15.30-17.45, 18:00-20:15, Jungius 9 Bibliothek AP, SemRm 3, 4, 5, 6, Beginn: 01.11.12
			& Prof. Dr. Ulrich Merkt, Prof. Dr. Ludwig Mathey, Daniela Pfannkuche, Thorsten Uphues, N.N. \\
\hline 66--103	& Tutorium zur Physik I und Einf"uhrung in die Theoretische Physik I
			& 2 SWS, keine Angaben (bei Studienbeginn unter www.physnet.uni-hamburg.de einsehbar), Hinweise in den Lehrveranstaltungen 66-100, 66-101, 66-102, Beginn: 3. Vorlesungswoche
			& Prof. Dr. Ulrich Merkt, Prof. Dr. Ludwig Mathey \\

% Nebenfach: Informatik
\hhline{|=|=|=|=|} 64--000 & Vorlesung Softwareentwicklung I
			& 2 SWS, Mi. 14:15-15:45, H"orsaal Chemie A, Beginn: 17.10.12
			& Dr. Axel Schmolitzky \\
\hline 64--001 & "Ubungen zu Softwareentwicklung I (7 Gruppen)
			& 2 SWS, Mo. 09:00-12:00, Di. 09:00-12:00, 14:00 - 17:00, Mi. 09:00-12:00, Do. 09:00-12:00, 14:00-17:00, Fr. 09:00-12:00, D-010, D-017, D-018, D-114, Beginn: 18.-24.10.12
			& Dr. Axel Schmolitzky, Christian Sp"ah, Till Aust, Philip Joschko, Arne Koors, Sven Magg, Fredrik Winkler, Claudia Wyrwoll, Christin Zahner \\
\hline 64-070 & Vorlesung Algorithmen und Datenstrukturen
			& 3 SWS, Mi. 10:15-11:45,Phil C, Fr. 14:15-15:45, Phil B, Beginn: 17.10.12
			& Prof. Dr. Matthias Rarey \\
\hline 64-071 & "Ubungen zu Algorithmen und Datenstrukturen (10 Gruppen)
			& 1 SWS, Mi. 12:15-13:45, F-009, Mi. 14:15-15:45, ZBH Rm 16, F-635, F-534, Mi. 16:15-17:45, D-129, F-534, Do. 10:15-11:45, F-534, F-334, Fr. 10:15-11:45, 12:15-13:45, ZBH Rm 16, Beginn: 19.-26.10.12
			& Mathias Michael von Behren, Matthias Hilbis, Matthias Kerzel, Florian Lauck, Thomas Wagner, N.N. \\

% Gibt es nicht mehr.
%\hline 64-080 & Vorlesung Grundlagen von Datenbanken (GDB) (Modul IP5)
%              & 3st. Mi 10:15-11:45, MLK6 HsB Chem Di 14:15 15:45, VMP8 Erzwiss H, Beginn: 21.10.09
%              & Norbert Ritter \\
%
%\hline 64-081 & "Ubung zu Grundlagen von Datenbanken (Modul IP5)
%              & 1st. Mo 10:15-11:45 Mo 12:15-13:45 Mo 14:15-15:45 Di 8:15-9:45 Di 10:15-11:45, Beginn 19.10.09
%              & Norbert Ritter, Marc Holze, Fabian Panse, Michael von Riegen \\
			&&&\\
\hline
\end{tabularx}

F"ur Informatik waren dies nur einige der zu empfehlenden Veranstaltungen.
Weitere Informatik-Vorlesungen
siehe \url{www.informatik.uni-hamburg.de/Info/Studium}.

\begin{tabularx}{\textwidth}{|l|X|X|X|}
% Nebenfach: VWL / BWL?
\hline 21-10.010 & Grundlagen des Rechnungswesens - Z1
			& 3 SWS, Di. 09:00 - 12:00, Audimax 1, Beginn: 16.10.12
			& Dr. Andreas Mammen \\
\hline 21-10.010 & "Ubungen zu Grundlagen des Rechnungswesens - Z1  (10 Gruppen)
			& 1 SWS, Mo. 13:00-14:00, WiWi B1, WiWi 3136/3142 (30 Plätze für Holzwirte), Di. 08:00-09:00, WiWi 2101/2105 (nur für Wi-Ing), WiWi 2098/2194, Di. 15:00-16:00, WiWi 2163/2168 (nur für Wi-Ing), WiWi 2091/2201, Mi. 16:00-17:00, WiWi 2067/2071, 2091/2201, Mi. 17:00-18:00, WiWi 2067/2071, 2091/2201, Beginn: 22.-24.10.12
			& Dr. Andreas Mammen \\
\hline 21-10.011 & Grundlagen des Rechnungswesens - Z2
			& 3 SWS, Mo 10:00-13:00, Audimax 1, Beginn: 15.10.12
			& Dr. Ralf Wi"smann \\
\hline 21-10.011 & "Ubungen zu Grundlagen des Rechnungswesens - Z2 (10 Gruppen)
			& 1 SWS, Mo. 09:00-10:00, WiWi 0029, WiWi 2163/2168, Di. 17:00-18:00, WiWi 0029, 2163/2168, Di. 18:00-19:00, WiWi 0029, 2163/2168, Fr. 08:00-09:00, WiWi 0079, WiWi 2067/2071 (nur für LL.bB. Arbeits- + Sozialmanagement und LL.B. Finanzen + Versicherung), Fr. 16:00
			-17:00, WiWi 2095/2197, 2101/2105, Beginn: 22.-23./26.10.12
			& Dr, Ralf Wi"smann \\
\hline 21-10.020 & Grundlagen der Wirtschaftsinformatik - Z1
			& 2 SWS, Mo. 10:00-12:00, Erzwiss H, Beginn: 15.10.12
			& Dr. Gabriele Schneidereit, Dr. Silvia Schwarze \\
\hline 21-10.020 & "Ubungen zu Grundlagen der Wirtschaftsinformatik - Z1 (6 Gruppen)
			& 2 SWS, Mo. 08:00-10:00, WiWi 2067/2071, 2091/2201, Do. 10:00-12:00 WiWi 2091/2201, 2095/2197, Fr. 14:00-16:00 WiWi 2091/2201, 2101/2105, Beginn: 22./25.-26.10.12
			& Dr. Gabriele Schneidereit, Dr. Silvia Schwarze \\
\hline 21-10.021 & Grundlagen der Wirtschaftsinformatik - Z2
			& 2 SWS, Di. 10:00-12:00, Erzwiss H, Beginn: 16.10.12
			& Dr. Kai Br"ussau, Dr. Gabriele Schneidereit \\
\hline 21-10.021 & "Ubungen zu Grundlagen der Wirtschaftsinformatik - Z2 (8 Gruppen)
			& 2 SWS., Di. 08:00-10:00, WiWi 0079, 2091/2201, Mi. 10:00-12:00, WiWi 2054/2055, 2091/2201, Mi. 16:00-18:00, WiWi 2054/2055, 2163/2168, Do. 12:00-14:00, WiWi 2091/2201, 2095/2197, Beginn: 23.-25.10.12
			& Dr. Kai Br"ussau, Dr. Gabriele Schneidereit, Dr. Robert Stahlbock, Stefan Lessmann \\

% Nebenfach: TUHH? (TODO)
%\hhline{|=|=|=|=|} TUHH & Grundlagen der Elektrotechnik I
%                        & Donnerstag, 11:30-13 in H - SBS95 Raum Audimax1
%                        & G"unter Ackermann \\
%
%&&&\\
%
%\hline TUHH & Elektrotechnik I / Grundlagen der Elektrotechnik I
%            & Montag, 8:00-9:30 in I - DE22 Raum Audimax2, Donnerstag, 11:30-12:45 in I - DE22 Raum Audimax2
%            & Jan-Luiken ter Haseborg \\%%\hline TUHH & Technische Mechanik I
%            & Mittwoch, 8:00-9:30 in I - DE22 Raum Audimax2
%            & Uwe Weltin \\
\hline
\end{tabularx}
\begin{tabularx}{\textwidth}{|l|X|X|X|}
\hline 21-10.095 & Makro"okonomik - Z1
			& 3 SWS, Fr. 09:00-12:00, Audimax 1, Beginn: 19.10.12
			& Dr. Sven Schreiber \\
\hline 21-10.095 & "Ubungen zu Makro"okonomik (8 Gruppen)
			& 1 SWS, Mo. 13:00-14:00, WiWi 2067/2071, 2091/2201, Mi. 09:00-10:00, WiWi 2101/2105, 2067/2071, Mi. 11:00-12:00, WiWi 2163/2168, 2095/2197, Fr. 13:00-14:00, WiWi 2091/2201, 2095/2197, Beginn: 22./24./26.10.12
			& Dr. Sven Schreiber \\
\hline 21-10.096 & Makro"okonomik - Z2
			& 3 SWS, Do. 8:00-11:00, Audimax 1, Beginn: 18.10.12
			& Prof. Dr. Lena Dräger \\
\hline 21-10.096 & "Ubungen zu Makro"okonomik - Z2 (8 Gruppen)
			& 1 SWS, Mo. 12:00-13:00, WiWi 2067/2071, 2091/2201, Mi. 08:00-09:00, WiWi 2101/2105, 2067/2071, Mi. 10:00-11:00, WiWi 2163/2168, 2095/2197, Fr. 12:00-13:00, WiWi 2091/2201, 2095/2197 (für Wirtschaft und Kultur Chinas u. LL. B. Arbeits- und Sozialmanagement), Beginn: 22./24./26.10.12
			& Prof. Dr. Lena Dräger \\
\hline 21-10.110 & Investition - Z1
			& 2 SWS, Do 16:00-18:00, ESA A, Beginn: 18.10.12
			& Prof. Dr. Martin Nell \\
\hline 21-10.110 & "Ubungen zu Investition - Z1 (8 Gruppen)
			& 2 SWS, Mo. 15:00-17:00, WiWi 2098/2194, 3136/3142, Di. 12:00-14:00, WiWi 2095/2197, 3136/3142 (beide für Wi-Ing), Mi. 12:00-14:00, WiWi 2098/2194, 3136/3142, Do. 08:00-10:00, WiWi 2098/2194, 2095/2197 (40 Plätze für LL.B. Finanzen und Versicherung), Beginn: 22.-25.10.12
			& Prof. Dr. Martin Nell \\
\hline 21-10.111 & Investition - Z2
			& 2 SWS, Do. 14:00-16:00, ESA A, Beginn: 18.10.12
			& Prof. Dr. Martin Nell \\
\hline 21-10.111 & "Ubungen zu Investition - Z2 (8 Gruppen)
			& 2 SWS, Mo. 08:00-10:00, 10:00-12:00, WiWi 2098/2194, 3136/3142, Mi. 15:00-17:00, WiWi 2098/2194, 2101/2105, Do. 12:00-14:00, WiWi 2067/2071, 2098/2194, Beginn: 22./24.-25.10.12
			& Prof. Dr. Martin Nell \\
\hline
\end{tabularx}
