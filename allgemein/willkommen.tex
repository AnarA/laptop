\subsection*{Herzlich Willkommen!}

\textbf{Definition:} Ein Mensch am Anfang seines Mathematikstudiums heiße
Studienanfänger bzw.  Studienanfängerin.

\textbf{Bemerkung:} Der allgemeine Sprachgebrauch gestattet es, eine(n)
Studienanfänger(in) auch als Erstsemest(l)er(in) (kurz: Ersti) zu bezeichnen.

Es ist anschaulich klar, dass ein Studienanfänger bzw. eine Studienanfängerin,
da er (sie) per Definitionem noch nicht studiert, nicht wissen kann, worum es
eigentlich geht. Daraus motiviert sich folgender

\textbf{Satz und Definition:} Ein Studienanfänger bzw. eine Studienanfängerin
benötigt eine Orientierungshilfe. Diese Orientierungshilfe heiße
Orientierungseinheit (kurz: OE).

\textbf{Beweis:} Der elementare Beweis sei dem/der geneigten Leser(in) als
leichte Übungsaufgabe selbst überlassen.

\textbf{Bemerkung:} Da die Sinnfälligkeit des obigen Satzes selbst
hanseatischen Behörden geläufig ist, genügt man mit der Teilnahme an der OE dem
im Hamburgischen Hochschulgesetz, \S 52, Absatz 2 geforderten Besuch einer
Studienfachberatung in den ersten beiden Studiensemestern.
