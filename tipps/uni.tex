\subsection{Was die Uni so alles zu bieten hat}

\subsubsection{Wo bitte geht's zum Rechnen?}

Zum Lernen und Arbeiten stehen im Geomatikum folgende studentische Arbeitsräume
zur Verfügung: im Erdgeschoss E24, im Techniker-Geschoss die Räume T27, T33,
T35 und T37 (und natürlich der FSR (T30), wobei dieser Raum in erster Linie ein
Aufenthaltsraum ist und es daher dort normalerweise etwas lauter ist), im 2.
Stock die Räume 242 und 244 und im 4. Stock der Raum 427. Außerdem dürft ihr
natürlich in allen Übungsräumen arbeiten, solange dort keine Veranstaltungen
sind. Desweiteren gibt es noch einen Gruppenraum in der Bibliothek, in der man
natürlich auch an den Lesetischen still arbeiten kann. Da das Geomatikum am
Samstag ab 13 Uhr und am Sonntag durchgehend geschlossen ist, bietet sich für
diese Zeit die Staatsbibliothek (Stabi), geöffnet Montag bis Freitag von 9-21,
Samstag von 10-18 und Sonntag von 12-18 Uhr, an. Dort gibt es Lesesäle zur
ruhigen Arbeit und Bereiche, in denen sich Gruppen unterhalten dürfen.

\subsubsection{Ich hab' Hunger!}

Schon bald werdet ihr unsere heimische Geomatikum-Mensa das erste Mal leid
sein. Dann empfiehlt sich ein Besuch auf dem Campus in einer der drei großen
Mensen im Philosophenturm, WiWi-Bunker oder der alten Hauptmensa hinter dem
Audimax. In den ersten beiden locken große Salatbuffets, eine Pastabar und eine
Wok-Station. Sehr gut ist auch die TU-Mensa in Harburg. Aber nach einiger Zeit
werdet ihr bestimmt des dauernden Laufens überdrüssig werden und reumütig ins
Geomatikum zurückkehren\ldots

\subsubsection{Schwitzen leicht gemacht}

Als Ausgleich zu der vielen kopflastigen Arbeit des Studiums eignet sich ein
Sport sehr. Über das Hochschulsportprogamm
(\url{http://www.hochschulsport-hamburg.de}), welches allen Hamburger Studenten
und -- gegen Aufpreis -- allen Mitarbeitern der Hochschulen und auch Externen
offensteht, erhaltet ihr ein umfassendes Angebot: von Rudern und Segeln auf der
Alster über alle gängigen Ballsportarten bis zu Kampfsport, Yoga, Bogenschießen
und die Teilnahme im Hochschulsport-Fitnessstudio -- hier findet jeder seine
Sportart.  Die Preise sind günstiger als in Vereinen.

\subsubsection{Qu\'{e} hablas?}

Wer mit seinem Studium noch nicht genug ausgelastet ist und seine
Fremdsprachenkenntnisse auffrischen oder ausbauen möchte, hat hierzu zwei
Möglichkeiten. Das Fachsprachenzentrum der Uni bietet Kurse auf hohem Niveau zu
verschiedenen Themengebieten an, in denen Fach- und Vokabelwissen kombiniert
werden. Zur Teilnahme muss man an einem Einstufungstest zu Beginn des Semesters
teilnehmen -- auf die neonfarbenen Plakate achten! Wer das erforderliche Niveau
nicht nachweisen kann, kann dieses in den Fremdsprachenkursen der
Volkshochschule in Kooperation mit der Universität erwerben. Neuerdings sind
diese Kurse kostenlos, eine frühzeitige Anmeldung ist daher zu empfehlen.
Informationen zu beiden Arten von Kursen findet ihr unter
\url{http://www.uni-hamburg.de/fremdsprachen.html}.

\subsubsection{Hilfe!}

Viele Studenten geraten im Laufe ihres Studiums mindestens einmal in eine Krise
und fragen sich dann: Studiere ich das Richtige? Bin ich überhaupt für die Uni
geeignet? Warum komme ich nicht mit? Manchmal kommen dann noch persönliche
Krisen und Probleme hinzu, die das Studium erschweren oder sogar unmöglich
machen. Solltet ihr irgendwann im Verlauf eures Studiums mit Problemen dieser
Art zu kämpfen haben, und helfen euch auch Gespräche mit Freunden und
Kommilitonen nicht weiter, dann legen wir euch das Zentrum für Studienberatung
und Psychologische Beratung der Universität wärmstens ans Herz (über
\url{http://www.uni-hamburg.de} $\rightarrow$ Studierende $\rightarrow$ Zentrum
für\ldots).  Neben einer kostenlosen, persönlichen und umfassenden Betreuung
und Beratung bietet das Zentrum auch Kurse und Workshops an, beispielsweise zur
Bekämpfung von Prüfungsangst, dem Sprechen vor einer Gruppe oder selbständigem
Arbeiten.

\subsubsection{Soft skills -- starke Fähigkeiten}

Das Zentrum für Studienberatung und Psychologische Beratung bietet auch andere
Kurse an, die auf eine Stärkung der sogenannten \emph{soft skills} abzielen.
Das sind unter anderem freies Sprechen, Lern- und Arbeitstechniken,
Stressbewältigung, Zeit- und Selbstmanagement. Auch das Career Center der
Universität ({\tt www.uni-hamburg.de} $\rightarrow$ Studierende $\rightarrow$
Career Center) bietet Kurse dieser Art, Vorträge und andere Veranstaltungen an,
die der Berufsorientierung dienen. In der Regel sind diese Veranstaltungen
kostenlos, eine Anmeldung ist für die Kurse jedoch erforderlich.

\subsubsection{Das liebe Geld}

Oft leiden Studenten unter chronischem Geldmangel. Die Universität und der
Fachbereich bieten durch zahlreichen Tutoren- und Hilfskraft-Tätigkeiten die
Möglichkeiten für kleine Zuverdienste an. So arbeiten Studenten etwa als
Aufsicht in der Bibliothek oder in den Computerräumen, und für die etwas
fortgeschrittenen Semester gibt es natürlich viele Jobs als Korrektor von
Übungsaufgaben, Übungshelfer oder sympathischer und munterer Tutor der
Orientierungseinheit. Auch an anderen Einrichtungen und Instituten der Uni gibt
es Jobs für Studenten. Haltet einfach Augen und Ohren offen und hört euch aktiv
um, indem ihr Professoren und Studenten höherer Semester ansprecht.
