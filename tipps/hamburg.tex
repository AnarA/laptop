\subsection{Was Hamburg so alles zu bieten hat \dots}

Im folgenden haben wir einige praxiserprobte Tipps zusammengestellt, mit denen
ihr eure Freizeit in dieser wunderbaren Stadt gestalten oder auch nur eure
Freistunden füllen könnt.

\subsubsection{Caf\'es}

% TODO: Trennung Sta-bi

Wem die Cafeteria im Geomatikum zu ungemütlich ist, findet auf dem Weg zum
Campus mehrere Caf\'e-Alternativen an der sogenannten \emph{Grindelmeile}.  Von
der Stabi bis zu den Grindelhochhäusern drängen sie sich dicht an dicht: je
nach Ausrichtung gibt es dann dazu z.B. Fußball live (Campus Caf\'e), Milchreis
in verschiedenen Varianten (Caf\'e R-Eis), viel Bio (Caf\'e Aramon), einen
Friseur (Gute Köpfe), preiswertes Frühstück (Caf\'e Backwahn) oder Lesungen und
Bücher-Stöber-Kisten (Caf\'e Mathilde (Ecke Beim Schlump/Bogenstraße) und Bar
Mathilde (beim Abaton-Kino), \url{http://www.mathilde-hh.de}).

\begin{advice}
Das Hadley's ist ganz nah am Geomatikum in einem ehemaligen Krankenhaus
untergebracht (Beim Schlump 84a). Im Sommer kann man im wunderschönen und
ruhigen Innenhof sitzen und Trinkschokolade in fünf Geschmacksrichtungen
genießen. Im Inneren locken gemütliche Sofas und Sessel.
\end{advice}

Wer's gerne hip mag, trinkt sein Koffein-Heißgetränk in der Schanze auf der
sogenannten \emph{Gal\~{a}o-Meile} (Schulterblatt und angrenzende Straßen).
Gegenüber vom Zentrum der linken Szene, der Roten Flora, kann man hier sehen
und gesehen werden und nach dem Milchkaffee noch trendig shoppen.

Bekannt für gutes Eis und ausgefallene Sorten sind die Filialen von Eiszeit
(Müggenkampstraße~45, Mühlenkamp~46 und Falkenried~47). Am nähesten beim
Geomatikum ist das Eiscaf\'e La Veneziana an der Grindelallee.

\begin{advice}
Bei Eisliebe in Altona (Bei der Reitbahn 2) gibt's das beste Eis, ein kleiner
Laden mit sehr vielen Sorten. Und wenn man schon mal in Altona ist, gleich noch
im Bonscheladen (Friedensallee 12) vorbeischauen, hier werden Bonbons selbst
gemacht.
\end{advice}

\subsubsection{Fast Food}

\begin{advice}
Die besten Döner gibt's bei Dubara: Jarrestraße~57, Wandsbeker Marktstraße~47,
Schlossmühlendamm (Harburg).
\end{advice}

Die nächsten McDonald's-Filialen sind am Dammtor-Bahnhof und an der U-Bahn
Hoheluft. An der Grindelallee gibt es die Sandwiches von Subways. Zahlreiche
Döner-Läden liegen ebenfalls auf der Strecke zwischen Campus und Geomatikum.
Außerdem sorgen viele asiatische Imbisse rund um die Kreuzung der Grindelallee
und Rentzelstraße für Abwechslung. Als einziger Pizzaservice liefert Joey's ins
Geomatikum. Eine Preisliste liegt im FSR (Raum T30).

\begin{advice}
Der Curry-Grindel (Renzelstraße~4) ist ein Muss für jeden, der einmal in
Hamburg ist. Hier gibt es eine große Auswahl an hervorragenden Curry-Würsten.
Dazu den berühmten Gurkensalat und das alkoholfreie Ingwer-Bier. Mit Hamburger
Hafen-Flair werden die Gerichte an den Mann gebracht. Das entschädigt auch die
moderaten Preise.
\end{advice}

\subsubsection{Essen und Trinken}

Die oben genannten Caf\'es bieten meist auch Snacks und Salate. Wer etwas mehr
braucht, wird vielleicht in der Geomatikum-Stammkneipe Geo 53 (gute Pizzen,
Mittagstisch, auch zum Mitnehmen; Beim Schlump 53), im Roxie (Burger,
Sandwiches und Co, mit Biergarten; Rentzelstraße~6) oder im Down Under
(berühmt-berüchtigt: Chicken Wings all you can eat; Ecke
Grindelallee/Bundesstraße) fündig.

\begin{advice}
Gute und bezahlbare Burger in netter Atmosphäre gibt's bei Doris Diner
(Grindelhof 43). Frühstücken kann man in aller Ruhe im Mangold Lokal (Ölmühle
30, U Feldstraße).
\end{advice}

Das Bistro des Abaton-Kinos (Allende-Platz) hat tolle Gerichte für faire
Preise. Im Sommer kann man man draußen sitzen. Wer nach dem Besuch des Holi-Kinos (Ecke
Schlankreye/Grindelberg) noch einen schnellen Happen braucht, sollte im kleinen
Senza Nome (Lehmweg 6) die \emph{Pizza Speck} probieren. Wenn's etwas mehr sein darf,
hat der große Nachbar Block House (Ecke Hoheluftchaussee/Lehmweg)
hervorragendes Fleisch. Sein etwas lässigerer Ableger ist Jim Block
(Fuhlsbüttler Str.~165, Kirchenallee 37), wo aus dem gleichen guten Fleisch
tolle Burger gemacht werden.

In der Schanze findet der spät aufgestandene Student viele Orte für ein spätes
Frühstück, z.B. in Omas Apotheke oder dem Frank und Frei (Ecke
Schanzenstraße/Susannenstraße). Ebenfalls in der Schanze gibt es gleich zwei
Ableger der Bok-Restaurants (Schulterblatt 3, Schanzenstraße~27), die für etwas
gehobenere Preise asiatisches Essen aus mehreren Ländern anbieten, von Sushi
bis Ente süß-sauer.

\begin{advice}
Am Allende-Platz gibt es einen Imbiss mit gefüllten Ofenkartoffeln (Kumpir) zum
studentengerechten Preis für \EUR{2 bis 4}, wobei für die Füllungen nicht nur
Sour Cream, sondern eine große Zahl von Zutaten zur Verfügung steht: Spinat,
Pepperoni, Mais, Champignons, Thunfisch, Schafskäse, verschiedene Soßen und
vieles mehr. Sattwerden garantiert!
\end{advice}

Ein paar Ecken weiter (Ecke Wohlwillstraße/Brigittenstraße) gibt es in der
Trattoria da Rocco alles, was das Italiener-Herz begehrt, inklusive einer
lautstarken \glqq Buena sera!\grqq-Begrüßung. Ebenfalls italienisch, aber in
modernem Ambiente und zum Zugucken wird im Vapiano (Ecke
Rothenbaumchaussee/Turmweg) gekocht. Die günstigen Preise von \EUR{5 bis 8}
locken mittags Büroleute und abends Harvestehudes Schickeria an. Das Altamira
in Ottensen (Thomasstraße) bietet eine große Auswahl an leckeren Tapas, hier
ist für jeden Geschmack etwas dabei. Im sogenannten \emph{Portugiesen-Viertel}
(die Straßen zwischen Michel und Elbufer) verbringt auch Otto Waalkes gerne den
Abend in einem der (meist Fisch-)Restaurants. Das Restaurant Mongolei
(Wandsbeker Zollstraße~155) hat für \EUR{12,80} täglich ein Büfett mit
zahlreichen asiatischen Gerichten zu bieten. Das Fleisch wählt man selbst aus
und bekommt es vor seinen Augen nach Wunsch gebraten.

Eine Cocktailbar mit großer Auswahl findet man im Gebäude der U-Bahn Mundsburg,
das Ambiente ist aber etwas kühl und die Preise für \EUR{6} normal bis hoch.
Auch im Down Under (s.o.) gibt es Cocktails (montags für \EUR{5,10}). In der
Turmbar des Hotel Hafen Hamburg (Landungsbrücken) ist auf die Cocktailpreise
die fantastische Aussicht draufgeschlagen.

\subsubsection{Weggehen}

Wenn im Frühjahr die ersten wärmeren Sonnenstrahlen locken, kurbelt der
Hamburger nicht nur sofort sein Cabrio-Verdeck herunter, sondern erfreut sich
seit einigen Jahren vor allem an den Beach Clubs, die an mehreren Standorten in
Hamburg, vor allem entlang des Elbufers hinter dem Fischmarkt, den Sommer über
aufmachen. In allen wird feiner Sand in großen Mengen ausgekippt, Liegestühle,
Hängematten, teilweise Sofas und Liegewiesen locken zum Verweilen, während man
den Containerschiffen auf der Elbe zuschaut. Bei entsprechenden Preisen gibt es
Cocktails und andere Getränke, manchmal auch Snacks oder Gegrilltes. Schaut
einfach im nächsten Frühling in die Zeitungen, wo es dieses Mal Clubs gibt.

\begin{advice}
Eine grandiose Cocktailbar ist das Cairos (Hoheluftchaussee~117). Der beste
Cocktail: der \emph{Pangalaktische Donnergurgler}. Danach zum Billardspielen
oder zum Kickern an Hamburgs längsten Kickertisch ins Queue (Fuhlsbüttler
Straße~184).
\end{advice}

Wer's etwas gemütlicher mag, findet in der Schanze rund um das Schulterblatt
dicht an dicht Locations, in denen man den Abend verbringen oder auch nur ein
wenig vorglühen kann. Am frühen Abend und bei entsprechendem Wetter werden auch
Bierbänke nach draußen und in die Hinterhöfe gestellt. Zu späterer Stunde
strandet man dann gemütlich in der Sofabar (Ecke Schanzenstraße/Beckstraße).

Unbestrittenes Herz des Hamburger Nachtlebens ist der Kiez. Das sind die
berühmt-berüchtigte \emph{Reeperbahn} und ihre Nebenstraßen, vor allem die
Große Freiheit, mitten in St.~Pauli gelegen. Für ausgedehntes Kneipen-Hopping
als Einstieg für einen gelungenen Party-Abend empfehlen sich die zahlreichen
kleinen Läden in der Straße Hamburger Berg, darunter Roschinsky's, Rosie's Bar,
der Blaue Peter und die Barbarabar. Auch rund um den Hans-Albers-Platz gibt es
diverse Kneipen und Bars.

\begin{advice}
In der Talstraße~22 gibt es die Bar 3-Zimmer-Wohnung. Die Theke befindet sich
in der Küche, gemütlich sitzen lässt es sich im Wohnzimmer. Im Schlafzimmer
wird eine Partie Playstation gespielt. Eine gemütlich eingerichtete Bar, mit
Büchern zum Stöbern und Kickertisch, zum einfach nur wohlfühlen -- und das
mitten auf dem Kiez!
\end{advice}

\subsubsection{Clubs}

Der Club mit der größten Tradition ist zweifelsohne die Große Freiheit (Große
Freiheit~36). Hier sind schon alle möglichen Größen des Musikgewerbes
aufgetreten, auch heute gibt es hier noch oft Konzerte. An den anderen Abenden
bietet die Freiheit meist eine bunte Mischung auf dem Main-Floor, im Keller
Alternative und härteren Rock und in der Galerie Latino-Musik. Gleich nebenan
liegt das Grünspan (Große~Freiheit~58), \glqq Hamburgs Rockcenter No.~1\grqq\ 
laut Eigenwerbung. Auch hier gibt es oft Konzerte.

Das Molotow (Spielbudenplatz~5) ist seit Jahren eine feste Größe auf dem Kiez.
Verschiedene DJs, Clubpartys und Events sorgen für Abwechslung, die Preise sind
vertretbar. Zum Aufwärmen eignet sich die Meanie Bar im selben Gebäude,
Eigenbeschreibung: \glqq Das Spektrum reicht musikalisch von französischen
Chansons und Beatperlen über schräge Experimental-Elektronik,
Sixties-Psychedelia bis hin zu russischer Folklore.\grqq

Der Golden Pudel Club direkt am Fischmarkt (Nr.~27) glänzt durch entspannte
Atmosphäre, ein interessant gemischtes Publikum, darunter auch immer wieder
Szenegrößen, und stimmige Musik von Elektro bis Hip-Hop, je nach DJ. An den
Wochenenden ist es oft brechend voll, im Sommer verlagert sich die chillende
Menge auf die angrenzenden Treppen mit Blick auf den Hafen.

Und wenn man schon einmal da ist, kann man auch gleich die Nacht durchmachen,
denn der unbestritten typischste Ausklang einer Party-Nacht führt einen am
Sonntagmorgen auf den \emph{Fischmarkt} (Große Elbstraße, nicht weit vom Kiez),
auf dem ab 4 Uhr echte Hamburger Originale wie \emph{Aale-Dieter} oder
\emph{Bananen-Benno} ihre Waren an den Mann bringen. Ein frisches Fischbrötchen
ist hier ein Muss. Um 7 Uhr, spätestens 8 Uhr ist der Zauber dann auch schon
wieder vorbei.

\subsubsection{Sonne tanken}

Gemütlich in der Sonne chillen kann an vielen Stellen in Hamburg, die Stadt
gilt nicht umsonst als eine der grünsten Städte Deutschlands. Besonders schön
hat man es auf den Alsterwiesen entlang des Harvestehuder Wegs mit Blick auf
die am Wochenende zahlreichen Jogger, Segler und Ruderer. Hier wird auch gerne
mal gegrillt. Auch am Gegenufer (Ostseite) gibt es ein paar grüne
Liegeflecken.

Der Elbstrand ist schon lange kein Geheimtipp mehr. Hier fühlt man sich fast
wie im Urlaub, nur dass eine beeindruckende Szenerie vorbeiziehender
Containerriesen hinzukommt. Zu Ostern werden an vielen Stellen große Lagerfeuer
entzündet und sorgen für eine schöne Stimmung. Grillen wird hier meist
toleriert, wenn man keinen Müll zurücklässt und die heiße Asche sicher
entsorgt. Leider ist der Övelgönner Strand etwas schwer zu erreichen, am besten
aber noch mit dem Bus 112 (von Bahnhof Altona) oder der Hafenfähre 62 (von
Landungsbrücken) bis Neumühlen. Ansonsten gibt es einige nur zu Fuß passierbare
Zugänge von der Elbchaussee aus (z.B. Himmelsleiter und Schulberg). Oder man
nähert sich vom anderen Ende, von Teufelsbrück aus.

Unter der Woche angenehm leer, am Wochenende brechend voll ist der Stadtpark
(zu erreichen über U Borgweg, U Saarlandstraße oder S Alte Wöhr, der Bus 179
hält vor dem Planetarium). Hier wird Fußball gespielt, jongliert, gegrillt oder
nur auf der Wiese entspannt, im Stadtparksee gibt es ein Freibad. Wer die
Eintrittskarten für die sommerlichen Open Air Konzerte auf der Freilichtbühne
(S Alte Wöhr) sparen möchte, kann auch von einer der angrenzenden Wiesen aus in
den vollen akustischen Genuss kommen.

\subsubsection{Anschauen}

Hamburg beherbergt zahlreiche Museen zu den verschiedensten Themen: Kunsthalle,
Völkerkundemuseum, Museum der Arbeit, Museum für Kunst und Gewerbe, Zoll- und
Gewürzmuseum und andere. Eine Anregung findet man unter
\url{http://www.freizeitziele.hamburg.de}.

\begin{advice}
Einmal im Jahr findet die \emph{Lange Nacht der Museen} statt. Einmal das
Ticket gezahlt, kann man die ganze Nacht alle Museen Hamburgs besuchen. Eine
ideale Möglichkeit, um überall einmal reinzuschnuppern, auch wenn es ziemlich
voll ist. Was einem gefallen hat, kann man ja später nochmal in Ruhe besuchen.
\end{advice}

Das Miniatur Wunderland (\url{http://www.miniatur-wunderland.de}) in der
Speicherstadt ist die größte Modelleisenbahn der Welt, eine unglaubliche
Vielfalt in Größe H0 bietet sich den Besuchern. Hier werden gestandene
Familienväter wieder zu Kindern und können sich nicht losreißen von den
Szenerien, die u.a. Hamburg, die Alpen, Amerika und Skandinavien darstellen und
mit Liebe zum Detail begeistern. Die Anlage wird ständig ausgebaut. Eine
Kartenreservierung empfiehlt sich, um Wartezeiten zu vermeiden.

Das Planetarium im Stadtpark (\url{http://www.planetarium-hamburg.de}) besitzt
eine der modernsten Projektionsanlagen weltweit und bietet ständig wechselnde
Veranstaltungen an, ein Besuch des ehemaligen Wasserturms lohnt sich also.

Der Hafen ist das Herzstück der Freien und Hansestadt Hamburg, die sich als Tor
zur Welt versteht. Zahlreiche Unternehmen bieten von den Landungsbrücken aus
Hafenrundfahrten an, die einen guten Überblick über das riesige Areal bieten.
Ergänzend kann man die Museumsschiffe Rickmer Rickmers
(\url{http://www.rickmer-rickmers.de}) und Cap San Diego
(\url{http://www.capsandiego.de}), beide an der Überseebrücke, besichtigen. Und
auch Hamburgs Wahrzeichen, die Hauptkirche St. Michaelis (Michel), ist nicht
weit entfernt (Ludwig-Erhard-Straße). Bei schönem Wetter hat man vom Turm aus
einen tollen Blick über die Stadt. Doch auch das Innere der Kirche ist einen
Blick wert. In der Adventszeit finden hier stimmungsvolle Weihnachtskonzerte
statt.

\subsubsection{Freizeit}

Das geflügelte Wort ist tatsächlich wahr: Hamburg hat wirklich mehr Brücken als
Venedig, und das hat auch seinen Grund, denn Hamburg ist durchzogen von kleinen
Wasserstraßen, Kanälen und Fleeten. Diese lassen sich hervorragend vom Wasser
aus erkunden, ob im Kanu, Tretboot oder Kajak, und erlauben beeindruckende
Einblicke in die Gärten der großen Winterhuder Villen. So kommt man von
Eimsbüttel bis Ohlsdorf, in den Stadtparksee und nach Barmbek -- und natürlich
auf die große Außenalster. Bootsvermieter finden sich überall an den Kanälen.
Ist man einmal unterwegs, empfiehlt sich ein Besuch des Caf\'e Canal am
Poelchaukamp: hier werden per Seilzug Kaffee und Kuchen ins Boot geliefert.

Im Winter kann man in den Wallanlagen (Sievekingplatz) Eislaufen, im Sommer auf
derselben Bahn rollerbladen. Nebenan auf dem Heiligengeistfeld (U Feldstraße
und U St. Pauli) findet drei mal im Jahr Hamburgs größtes Volksfest, der DOM,
statt. Zahlreiche Fressbuden und Fahrgeschäfte laden zum Bummeln ein.  Leider
sind von den traditionellen Attraktionen über die Jahre immer mehr
verschwunden, und das Publikum ist prolliger geworden; das Riesenrad hält sich
jedoch und erlaubt einen schönen Blick über das nächtliche Hamburg. Freitags
gibt's Feuerwerk, Mittwoch Familientag mit etwas günstigeren Preisen. Pflicht:
Einmal \emph{Hau den Lukas} spielen (immer an der Achterbahn).

\begin{advice}
Einmal im Monat könnt ihr im Rabatzz Spielen was das Zeug hält! An jedem
dritten Donnerstag im Monat ist Ü18-Abend. Von 19 bsi 23 Uhr kann man sich hier
mal so richtig austoben, Trampolin und Hüpfburg springen, im fast freien Fall
die Rutsche runterflitzen und ganz viel mehr. Das Rabatzz findet ihr in der
Kieler Straße~571, es lohnt sich, ganz früh, bereits um 19 Uhr, da zu sein!
\end{advice}

Ohne Fahrgeschäfte geht's auf dem Alstervergnügen, das jährlich drei Tage lang
rund um die Binnenalster stattfindet. Krönung ist jeden Abend das Feuerwerk, zu
dem jedes Jahr andere Nationen eingeladen werden. Frühes Platzsuchen ist
unerlässlich. Wem die Fressbuden und Saufläden auch noch zu viel sind, kann
beim Kirschblütenfest (jährlich im April/Mai) ein vom japanischen Konsulat
ausgerichtetes Feuerwerk auf der Außenalster genießen. Auch hier sollte man
sich frühzeitig einen guten Aussichtsplatz suchen.

\subsubsection{Kultur}

Hamburg macht viel Theater: Die großen Häuser Thalia Theater
(\url{http://www.thalia-theater.de}) und Schauspielhaus
(\url{http://www.schauspielhaus.de}) locken mit hohem Niveau und oft für
Studenten kräftig ermäßigten Preisen. Zahlreiche kleinere Theater wie der
Thalia-Ableger in der Gaußstraße (Ottensen), das St.  Pauli Theater, Hamburgs
ältestes Theater mit einer ruhmreichen Geschichte und direkt auf dem Kiez
gelegen (\url{http://www.st-pauli-theater.de}), das Ernst-Deutsch-Theater
(\url{http://www.ernst-deutsch-theater.de}) oder das English Theatre
(\url{http://www.englishtheatre.de}) ergänzen das Angebot. Daneben gibt es noch
Kabarett-Häuser und kleine Stadtteiltheater.

Für Kino-Freunde gibt es ebenfalls ein reichhaltiges Angebot: die großen
Blockbuster findet man im Cinemaxx am Bahnhof Dammtor und im Streit's
(Jungfernstieg), welches als einziges Hamburger Kino die großen Filme auch in
der Originalfassung zeigt. Besonders beliebt: die englische Sneak-Preview jeden
Montag; rechtzeitig um Karten kümmern! Direkt am Hauptcampus liegt das Abaton,
welches eher kleinere künstlerische Filme zeigt, oft auch als Original mit
Untertiteln; Reservierung empfohlen!

Das Holi hat den schönsten Leinwand-Vorhang in Hamburg und zeigt auch eher
Filme aus der zweiten Reihe; hier gibt's keine Platzkarten, rechtzeitig kommen!
Das Streit's am Jungfernstieg lockt mit einer Cocktailbar und einem Kinosaal
mit Parkett und Rang. Im Passage an der Mönckebergstraße findet man -- neben
zwei normal großen -- den kleinsten Kinosaal in Hamburg mit knapp 20 Plätzen.
Im Sommer gibt es auch immer wieder an mehreren Orten Freiluft-Kinos, teils
kostenpflichtig. Das aktuelle Hamburger Kinoprogramm, eine Vorschau auf die
nächste Woche und alle Adressen und Telefonnummern sowie alle Infos zu den Kinos
findet man auf \url{http://www.kino-fahrplan.de}.

Daneben gibt es noch das Uni-Kino im Audimax. Hier werden während der
Vorlesungszeit jeden Donnerstag ein oder zwei Filme, meist sehr aktuell, für
kleines Geld gezeigt. Anfang Dezember gibt es immer mehrere Vorführungen der
\emph{Feuerzangenbowle}. Mit mitgebrachten Keksen und Alkohol wird dieses
Ereignis kultig zelebriert. Auch hier gilt: rechtzeitig um Karten kümmern!

\begin{advice}
In den Zeisekinos gibt es nicht nur schöne Filme, sondern auch sogenannte
\emph{Slams}: Singer Slam (jeden 1.), Poetry Slam (jeden 2.) und Short Film
Slam (jeden 3. Freitag im Monat).
\end{advice}
