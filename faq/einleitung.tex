\subsubsection{Der -- Die -- Das}

\dots\ Mathestudenten sitzen meist nur in der Vorlesung oder Übung und hören
zu.  Anscheinend verstehen sie alles, haben keine Fragen und keine Interessen,
die der Vertretung nach außen wert wären.

\subsubsection{Wer -- Wie -- Was}

\dots\ ist denn da so an die Uni gekommen? Der Crack, der immer in der ersten
Reihe sitzt, ungehalten reagiert, wenn jemand den Prof mit Fragen aufhält und
scheinbar alles versteht.  Die Schöne, die sich täglich mindestens drei dumme
Sprüche anhören muss. Der Nerd, der immer seinen Laptop dabei hat. Die
Dozentin, die irgendwie zwei Meter über unserer Welt in Formeln schwebt. Die
beiden coolen Typen, die in der letzten Reihe über die gestrige Fete labern.
Der Kerl mit dem Rucksack, der immer eine Viertelstunde zu spät kommt. Die Type
mit dem Attach\'eeköfferchen (schweinsledern), die in der FAZ blättert.  Ich?

\subsubsection{Wieso -- Weshalb -- Warum}

\dots\ studieren wir eigentlich alle Mathe? \dots\ gibt es kaum eine
Professorin oder Dozentin am Fachbereich? \dots\ bin ich immer so schweigsam?
\dots\ frage ich so selten, obwohl mir doch so vieles unklar ist? \dots\ fällt
es mir so schwer, einfach meine Nachbarin zu fragen? \dots\ sitze ich jede
Woche alleine vor den Aufgaben? \dots\ geben 60 Prozent das Mathestudium auf,
bevor sie die ersten zwei Semester hinter sich haben? \dots\ kümmert sich hier
niemand um mich? \dots\ bin ich hier allen so egal?

\subsubsection{Wer nicht fragt bleibt dumm!}

So viele Fragen verlangen nach einer Antwort: 42! Aber hier wäre wohl doch eine
etwas konkretere Antwort angebracht. Leider kommt jetzt die große Enttäuschung:
In diesem Fall gibt es keine einzige oder richtige Antwort. Jeder entscheidet
einige Fragen für sich selbst, übernimmt andere aus dem Topf der allgemeinen
Meinung, tut sie als uninteressant ab oder macht sonst was mit seinem Wissen.
Warum denn nun die persönliche Meinung so oder so ausfällt, lässt sich ja auch
nicht immer begründen.

Die Universität ist ein derartiges Sammelsurium unterschiedlicher An- und
Einsichten -- was gerade am Fachbereich Mathematik trotz seiner geringen Größe
sehr ausgeprägt ist --, dass es der einzelnen Person nicht immer leicht fällt,
durch Wortgewandtheit und cooles Auftreten der anderen nicht irritiert oder
verunsichert zu werden. Das erfährt jeder mal am eigenen Leib, das ist kein
Bein- und schon gar kein Halsbruch. Ein Haken ist nur: Viele Neuankömmlinge
lassen sich schon in der OE dermaßen überrollen, dass sie kaum den Mund
aufbekommen. Die Vielfalt und Menge der neuen Eindrücke aus Uni und Umgebung
ist ja auch wahrlich unüberschaubar, gerade wenn eventuell auch noch der Umzug
in eine neue Stadt als Eindrucksquelle hinzukommt.

Da es auch immer wieder so ein paar unglaublich selbstsichere Hasskappen gibt,
die sich unter den merkwürdigsten Umständen weltgewandt und unbeeindruckt geben
können, bleibt meist die eigene Meinung ungehört. Unsinnigerweise sind es
nämlich gerade diese Personen, die als Orientierungshilfe dienen; meist ziehen
die auch nur eine gekonnte Show ab. Es soll ja Leute geben, die das
Fotografiertwerden vor dem Spiegel lernen.

Wichtig und studienbegleitend sollte folgender Merksatz sein: Keine Frage ist
zu dumm, um gestellt zu werden. Der feste Glaube, dass außer einem selbst
keiner diese Frage stellen würde, entpuppt sich fast immer als Irrglaube.
Sowieso sind die meisten Mitstudis für jede tempobremsende Frage überaus
dankbar. Es gibt kaum etwas Besseres für eine Vorlesung oder Übungsgruppe als
einer, der jede Behauptung hinterfragt. Bei einem Studium ist der Dialog
zwischen Lehrenden und Lernenden schließlich einer der Grundpfeiler für den
Erfolg.

Hinzu kommt, dass die Professoren sich über jede Bemerkung zu ihrer Tätigkeit
freuen werden. Erfahren sie doch auf diese Weise am ehesten etwas über ihre Art
zu unterrichten.

Und im Allgemeinen sind auch Professoren über so ein \glqq profanes\grqq\
menschliches Bedürfnis, wie es eine Reaktion nun einmal darstellt, nicht
erhaben!
