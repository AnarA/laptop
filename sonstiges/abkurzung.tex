\scriptsize

\begin{tabular}{|p{10mm} p{68mm}|}
\hline
AG     	& Arbeitsgemeinschaft/-gruppe \\
Aküvz  	& Abkürzungsverzeichnis \\
AP     	& Allende-Platz (am Grindelhof) \\
AS     	& Akademischer Senat \\
AStA   	& Allgemeiner Studierendenausschuss \\
BAföG 	& Bundesausbildungsförderungsgesetz \\
BaMa		& Bachelor-MathematikerIn \\
BdMN    & Bibliothek der Mathematik und\\
       	& Naturwissenschaften \\
Bem.   	& Bemerkung \\
Bib			& Bibliohek \\
B.Sc.		& Bachelor of Science \\
BMGN		& Bibliothek für Mathematik und Geschichte der Naturwissenschaften \\
BWL    	& Betriebswirtschaftslehre \\
bzw.   	& beziehungsweise \\
c.t.   	& cum tempore (mit akadem. Viertel) \\
cosh   	& hyperbolischer Kosinus \\
Def.   	& Definiton \\
d.h.   	& das heißt \\
DiMa   	& DiplommathematikerIn \\
Doz    	& DozentIn \\
ESA    	& Edmund-Siemers-Allee (Universitäts\-haupt\-gebäude am Dammtor) \\
etc.   	& et cetera (und weitere) \\
FA			& Fachausschuss \\
Fak			& Fakultät \\
FB     	& Fachbereich \\
FBR    	& Fachbereichsrat \\
FSR    	& Fachschaftsrat (oder -raum = T30) \\
FSRK   	& Fachschaftsrätekonferenz \\
FSZ    	& Fachsprachenzentrum \\
GEZ    	& Gebühreneinzugszentrale \\
GruMi  	& Studierende(r) des Lehramts der Grund- und Mittelstufe \\
HOE    	& Hauptstudiums-OE \\
H1-H6  	& Hörsäle des Geomatikums \\
HSG    	& Hochschulgruppe (meist ein politischer Ableger einer \glqq real existierenden\grqq\ Partei) \\
HVV    	& Hamburger Verkehrsverbund \\
HWP     & Hochschule für Wirtschaft und Politik \\
IGN    	& Institut für Geschichte der Naturwissenschaften, Mathematik und Technik \\
IZHD   	& Interdisziplinäres Zentrum für Hochschuldidaktik \\
KoMa   	& Konferenz der deutschsprachigen Mathematikfachschaften \\
KVV    	& Kommentiertes Vorlesungsverzeichnis \\
L\&L    & Lehr- und Lernformen \\
Lem.   	& \raisebox{0pt}[2.65mm]{Lemma (kleiner math. Nebensatz)} \\
\hline
\end{tabular}

\begin{tabular}{|p{10mm} p{68mm}|}
\hline
\LaTeX 	& [latäch] Textsatzprogramm \\
lim     & Limes \\
ln      & natürlicher Logarithmus \\
LOA    	& Stud. des Lehramts der Oberstufe an Allgemeinbildenden Schulen \\
LOB    	& Stud. des Lehramts der Oberstufe an Berufsbildenden Schulen \\
LPO    	& Lehrerprüfungordnung \\
M\&G    & Mathematik und Gesellschaft \\
max.    & maximal \\
min.    & mindestens; minimal; Minuten \\
MLKP   	& Martin-Luther-King-Platz (zw. Chemie und Zoologie) \\
NuMa    & (Nur-)MathematikerIn (=DiMa) \\
N.N.		& nomen nominandum (\textit{lat.} zu nennender Name) \\
o.ä.  	& oder ähnlich \\
OE     	& Orientierungseinheit \\
Philturm & Philosophenturm \\
PI     	& (Gebäude des) Fachbereichs Erzieh\-ungs\-wissenschaften \\
POE    	& Problemorientierte Einführung \\
q.e.d. 	& quod erat demonstrandum (was zu beweisen war) \\
SIV    	& Studentische Interessensvertretung \\
SoSe   	& Sommersemester (auch SS) \\
SP     	& Schwerpunkt \\
SP AD  	& SP Analysis \& Differentialgeometrie \\
SP AZ  	& SP Algebra \& Zahlentheorie \\
SP DD  	& SP Differentialgleichungen \& Dynamische Systeme \\
SP GD  	& SP Geometrie \& Diskrete Mathematik \\
SP GN  	& SP Geschichte der Naturwissenschaften, Mathematik und Technik \\
SP OA  	& SP Optimierung \& Approximation \\
SP ST  	& SP Mathematische Statistik \& Stochastische Prozesse \\
SPR   	& Schwerpunktsrat\\
s.t.		& sin tempore (ohne akadem. Viertel) \\
StaBi  	& Staatsbibliothek \\
StuPa   & Studierendenparlament \\
SWS    	& Semesterwochenstunde(n) \\
tanh   	& hyperbolischer Tangens \\
TeMa   	& TechnomatematikerIn \\
\TeX   	& [täch] Textsatzprogramm \\
TVP    	& Technisches und Verwaltungspersonal \\
u.     	& und \\
u.U.   	& unter Umständen \\
Übi   	& ÜbungsgruppenleiterIn \\
ÜG			& Übungsgruppe \\
UKE    	& Universitätklinikum Eppendorf \\
usw.   	& und so weiter \\
\hline
\end{tabular}

\begin{tabular}{|p{10mm} p{68mm}|}
\hline
VMP    	& Von-Melle-Park (= Campusgelände) \\
Vorl.		& Vorlesung \\
VV     	& Vollversammlung; Vorlesungsverzeichnis \\
VWL    	& Volkswirtschaftslehre \\
WiMa   	& WirtschaftsmathematikerIn \\
WiSe   	& Wintersemester (auch WS) \\
WiWi   	& Wirtschaftswissenschaften (auch WiWi-Bunker) \\
z.B.   	& zum Beispiel \\
ZBMI   	& Zentralbibliothek der Mathematischen Institute\\
ZMP			& Zentrum für Mathematische Physik \\
ZMS    	& Zentrum für Modellierung \& Simulation \\
zw.    	& zwischen \\
\hline
\end{tabular}

\begin{tabular}{|llp{21.75mm}|llp{21.75mm}|}

\hline
\multicolumn{6}{|c|}{Griechische Buchstaben} \\
\hline

\textit{A}       & $\alpha$                   & Alpha
    & \textit{N} & $\nu$                      & Nü       \\
\textit{B}       & $\beta$                    & Beta
    & $\Xi$      & $\xi$                      & Xi       \\
$\Gamma$         & $\gamma$                   & Gamma
    & \textit{O} & \textit{o}                 & Omikron  \\
$\Delta$         & $\delta$                   & Delta
    & $\Pi$      & $\pi$,      $\varpi$       & Pi       \\
\textit{E}       & $\epsilon$, $\varepsilon$  & Epsilon
    &\textit{P}  & $\rho$,     $\varrho$      & Rho      \\
\textit{Z}       & $\zeta$                    & Zeta
    & $\Sigma$   & $\sigma$,   $\varsigma$    & Sigma    \\
\textit{H}       & $\eta$                     & Eta
    & \textit{T} & $\tau$                     & Tau      \\
\textit{T}       & $\theta$,   $\vartheta$    & Theta
    & \textit{Y} & $\upsilon$                 & Ypsilon  \\
\textit{I}       & $\iota$                    & Iota
    & $\Phi$     & $\phi$,     $\varphi$      & Phi      \\
\textit{K}       & $\kappa$,   $\varkappa$    &Kappa
    & \textit{X} & $\chi$                     & Chi      \\
$\Lambda$        & $\lambda$                  & Lambda
    & $\Psi$     & $\psi$                     & Psi      \\
\textit{M}       & $\mu$                      & Mü
    & $\Omega$   & $\omega$                   & Omega    \\
\hline
\multicolumn{6}{|c|}{Hebräische Buchstaben} \\
\hline

$\aleph$            && Aleph   & {\fonti\char'151} && Beth    \\
{\fonti\char'152}   && Gimel   & {\fonti\char'153} && Daleth  \\

\hline
\multicolumn{6}{|c|}{Deutsche Buchstaben (Sütterlin)} \\
\hline

{\suetterlin A} & {\suetterlin a} & A
    & {\suetterlin N} & {\suetterlin n} & N \\
{\suetterlin B} & {\suetterlin b} & B
    & {\suetterlin O} & {\suetterlin o} & O \\
{\suetterlin C} & {\suetterlin c} & C
    & {\suetterlin P} & {\suetterlin p} & P \\
{\suetterlin D} & {\suetterlin d} & D
    & {\suetterlin Q} & {\suetterlin q} & Q \\
{\suetterlin E} & {\suetterlin e} & E
    & {\suetterlin R} & {\suetterlin r} & R \\
{\suetterlin F} & {\suetterlin f} & F
    & {\suetterlin S} & {\suetterlin s, \char255} & S, ß \\
{\suetterlin G} & {\suetterlin g} & G
    & {\suetterlin T} & {\suetterlin t} & T \\
{\suetterlin H} & {\suetterlin h} & H
    & {\suetterlin U} & {\suetterlin u} & U \\
{\suetterlin I} & {\suetterlin i} & I
    & {\suetterlin V} & {\suetterlin v} & V \\
{\suetterlin J} & {\suetterlin j} & J
    & {\suetterlin W} & {\suetterlin w} & W \\
{\suetterlin K} & {\suetterlin k} & K
    & {\suetterlin X} & {\suetterlin x} & X \\
{\suetterlin L} & {\suetterlin l} & L
    & {\suetterlin Y} & {\suetterlin y} & Y \\
{\suetterlin M} & {\suetterlin m} & M
    & {\suetterlin Z} & {\suetterlin z} & Z \\
\hline \end{tabular}

\vspace{3.5mm}
\begin{tabular}{|l|p{75.85mm}|}
\hline
$\mathbb{N}$ & natürliche Zahlen ($\mathbb{N}_0$: mit 0; $\mathbb{N}^\ast$: ohne 0)\\
$\mathbb{Z}$ & ganze Zahlen ($\mathbb{Z}_+$: alle $>0$; $\mathbb{Z}_-$: alle $<0$)\\
$\mathbb{Q}$ & rationale Zahlen\\
$\mathbb{R}$ & reelle Zahlen\\
$\mathbb{C}$ & komplexe Zahlen\\
\hline
\end{tabular}

\normalsize
